\documentclass[12pt, titlepage]{article}

\usepackage{amsmath, mathtools}
\usepackage[round]{natbib}
\usepackage{amsfonts}
\usepackage{amssymb}
\usepackage{graphicx}
\usepackage{colortbl}
\usepackage{xr}
\usepackage{hyperref}
\usepackage{longtable}
\usepackage{xfrac}
\usepackage{tabularx}
\usepackage{float}
\usepackage{siunitx}
\usepackage{booktabs}
\usepackage{multirow}
\usepackage[section]{placeins}
\usepackage{caption}
\usepackage{fullpage}
\usepackage{array}
\usepackage[nottoc,numbib]{tocbibind}
\usepackage{adjustbox}

\hypersetup{
	bookmarks=true,         % show bookmarks bar?
	colorlinks=true,        % false: boxed links; true: colored links
	linkcolor=red,          % color of internal links (change box color with linkbordercolor)
	citecolor=blue,         % color of links to bibliography
	filecolor=magenta,      % color of file links
	urlcolor=cyan           % color of external links
}


\externaldocument{../../SRS/SRS}

%% Comments

\usepackage{color}

\newif\ifcomments\commentstrue %displays comments
%\newif\ifcomments\commentsfalse %so that comments do not display

\ifcomments
\newcommand{\authornote}[3]{\textcolor{#1}{[#3 ---#2]}}
\newcommand{\todo}[1]{\textcolor{red}{[TODO: #1]}}
\else
\newcommand{\authornote}[3]{}
\newcommand{\todo}[1]{}
\fi

\newcommand{\wss}[1]{\authornote{blue}{SS}{#1}} 
\newcommand{\plt}[1]{\authornote{magenta}{TPLT}{#1}} %For explanation of the template
\newcommand{\an}[1]{\authornote{cyan}{Author}{#1}}

%% Common Parts

\newcommand{\progname}{Sayyara}
\newcommand{\authname}{Team 3, Tiny Coders
	\\ Arkin Modi
	\\ Joy Xiao
	\\ Leon So
	\\ Timothy Choy} % AUTHOR NAMES

\usepackage{hyperref}
\hypersetup{colorlinks=true, linkcolor=blue, citecolor=blue, filecolor=blue,
	urlcolor=blue, unicode=false}
\urlstyle{same}

\usepackage{parskip}
\usepackage{geometry}
\geometry{a4paper, portrait, margin=1in}


\begin{document}

\title{Module Interface Specification for \progname{}}

\author{\authname}

\date{\today}

\maketitle

\pagenumbering{roman}

\section{Revision History}

\begin{table}[hp]
	\caption{Revision History} \label{TblRevisionHistory}
	\begin{tabularx}{\textwidth}{llX}
		\toprule
		\textbf{Date}     & \textbf{Developer(s)} & \textbf{Change}                          \\
		\midrule
		December 28, 2022 & Arkin Modi            & Create Revision History                  \\
		January 7, 2023   & Joy Xiao              & Introduction                             \\
		January 9, 2023   & Arkin Modi            & Add Module Hierarchy                     \\
		January 11, 2023  & Arkin Modi            & Create MIS of Users Module               \\
		January 14, 2023  & Arkin Modi            & Create MIS of Work Orders Module         \\
		January 13, 2023  & Arkin Modi            & Create MIS of Database Driver Module     \\
		January 13, 2023  & Arkin Modi            & Create MIS of Services Module            \\
		January 15, 2023  & Arkin Modi            & Create MIS of Appointments Module        \\
		January 15, 2023  & Leon So               & Create MIS of Shop Module                \\
		January 15, 2023  & Arkin Modi            & Create MIS of Quotes Module              \\
		January 15, 2023  & Joy Xiao              & Create MIS of Employee Management Module \\
		January 17, 2023  & Timothy Choy          & Add Shop Lookup to Shop Module           \\
		March 1, 2023     & Arkin Modi            & Update Quote Request Bodies              \\
		March 7, 2023     & Timothy Choy          & Update Shop Module MIS                   \\
		March 14, 2023    & Arkin Modi            & Update Appointments Module               \\
		March 23, 2023    & Joy Xiao              & Update Employee Management Module        \\
		March 26, 2023    & Arkin Modi            & Remove createShop from Shop Module       \\
		April 2, 2023     & Timothy Choy          & Update createShopOwner from User Module  \\
		April 3, 2023     & Timothy Choy          & Update getShops for location             \\
		\bottomrule
	\end{tabularx}
\end{table}

~\newpage

\section{Symbols, Abbreviations and Acronyms}

See SRS Documentation at
\url{https://github.com/arkinmodi/project-sayyara/blob/main/docs/SRS/SRS.pdf}

% \wss{Also add any additional symbols, abbreviations or acronyms}

\begin{tabular}{l l}
	\toprule
	\textbf{symbol} & \textbf{description}                \\
	\midrule
	\progname       & The name of the program being built \\
	MIS             & Module Interface Specifications     \\
	MG              & Module Guide                        \\
	\bottomrule
\end{tabular}

\newpage

\tableofcontents

\newpage

\listoftables

\listoffigures

\newpage

\pagenumbering{arabic}

\section{Introduction}

The following document details the Module Interface Specifications for project Sayyara. Sayyara is
a progressive web application (PWA) which will act as a single platform for independent auto repair
shops and vehicle owners. This platform will allow independent auto repair shops and vehicle owners
to interact in a more efficient and effective manner. Vehicle owners can search for auto repair
shops and services; request quotes for service; book, view, and manage service appointments. On the
application, auto repair shop owners will be able to manage a list of employees; manage a list of
service types and corresponding service appointment availabilities; manage store information such
as location, hours of operation, and contact information. Auto repair shop owners and employees
will be able to manage quotes, service appointments, and work orders from a single application. The
MIS will detail specifications for the project described above.

Complementary documents include the System Requirement Specifications and Module Guide. The full
documentation and implementation can be found at
\url{https://github.com/arkinmodi/project-sayyara/}.

\section{Notation}

% \wss{You should describe your notation.  You can use what is below as
% 	a starting point.}

The structure of the MIS for modules comes from \citet{HoffmanAndStrooper1995}, with the addition
that template modules have been adapted from \cite{GhezziEtAl2003}. The mathematical notation comes
from Chapter 3 of \citet{HoffmanAndStrooper1995}. For instance, the symbol := is used for a
multiple assignment statement and conditional rules follow the form $(c_1 \Rightarrow r_1 | c_2
	\Rightarrow r_2 | ... | c_n \Rightarrow r_n )$.

The following table summarizes the primitive data types used by \progname.

\begin{center}
	\renewcommand{\arraystretch}{1.2}
	\noindent
	\begin{tabular}{l l p{7.5cm}}
		\toprule
		\textbf{Data Type} & \textbf{Notation} & \textbf{Description}                                             \\
		\midrule
		character          & char              & a single symbol or digit                                         \\
		integer            & $\mathbb{Z}$      & a number without a fractional component in (-$\infty$, $\infty$) \\
		natural number     & $\mathbb{N}$      & a number without a fractional component in [1, $\infty$)         \\
		real               & $\mathbb{R}$      & any number in (-$\infty$, $\infty$)                              \\
		real positive      & $\mathbb{R}+$     & any number in [0, $\infty$)                                      \\
		\bottomrule
	\end{tabular}
\end{center}

\noindent
The specification of \progname \ uses some derived data types: sequences, strings, and
tuples. Sequences are lists filled with elements of the same data type. Strings
are sequences of characters. Tuples contain a list of values, potentially of
different types. In addition, \progname \ uses functions, which
are defined by the data types of their inputs and outputs. Local functions are
described by giving their type signature followed by their specification.

\section{Module Decomposition}

The following table is taken directly from the Module Guide document for this project.

\begin{table}[H]
	\centering
	\begin{tabular}{p{0.3\textwidth} p{0.6\textwidth}}
		\toprule
		\textbf{Level 1}                                      & \textbf{Level 2}           \\
		\midrule

		Hardware-Hiding Module                                & ~                          \\

		\midrule

		\multirow{7}{0.3\textwidth}{Behaviour-Hiding Module}  & Users Module               \\
		                                                      & Quotes Module              \\
		                                                      & Appointments Module        \\
		                                                      & Work Orders Module         \\
		                                                      & Employee Management Module \\
		                                                      & Services Module            \\
		                                                      & Shop Module                \\
		\midrule

		\multirow{1}{0.3\textwidth}{Software Decision Module} & Database Driver Module     \\

		\bottomrule
	\end{tabular}
	\caption{Module Hierarchy}
	\label{TblMH}
\end{table}

\newpage

\section{MIS of Database Driver Module} \label{mDBDriver}
\subsection{Module}

schema.prisma

\subsection{Uses}

None

\subsection{Syntax}

\subsubsection{Exported Constants}

None

\subsubsection{Exported Types}

\textbf{Employee}

\begin{table}[H]
	\begin{tabular}{|p{0.25\textwidth}|p{0.25\textwidth}|p{0.40\textwidth}|}
		\hline
		\textbf{Output Name} & \textbf{Output Type}  & \textbf{Description}                        \\
		\hline
		id                   & String                & ID of Employee                              \\
		\hline
		create\_time         & String                & Create Time of Employee Account             \\
		\hline
		update\_time         & String                & Update Time of Employee Account             \\
		\hline
		first\_name          & String                & First Name of Employee                      \\
		\hline
		last\_name           & String                & Last Name of Employee                       \\
		\hline
		phone\_number        & String                & Phone Number of Employee                    \\
		\hline
		email                & String                & Email of Employee                           \\
		\hline
		password             & String                & Password of Employee's Account              \\
		\hline
		type                 & String                & Type of User                                \\
		\hline
		shop                 & Shop                  & Shop which Employee is registered under     \\
		\hline
		appointments         & List$<$Appointment$>$ & Appointments assigned to the Employee       \\
		\hline
		status               & String                & Employee status (i.e., Active or Suspended) \\
		\hline
	\end{tabular}
\end{table}

\textbf{Customer}

\begin{table}[H]
	\begin{tabular}{|p{0.25\textwidth}|p{0.25\textwidth}|p{0.40\textwidth}|}
		\hline
		\textbf{Output Name} & \textbf{Output Type}  & \textbf{Description}                  \\
		\hline
		id                   & String                & ID of Customer                        \\
		\hline
		create\_time         & String                & Create Time of Customer Account       \\
		\hline
		update\_time         & String                & Update Time of Customer Account       \\
		\hline
		first\_name          & String                & First Name of Customer                \\
		\hline
		last\_name           & String                & Last Name of Customer                 \\
		\hline
		phone\_number        & String                & Phone Number of Customer              \\
		\hline
		email                & String                & Email of Customer                     \\
		\hline
		password             & String                & Password of Customer's Account        \\
		\hline
		type                 & String                & Type of User                          \\
		\hline
		appointments         & List$<$Appointment$>$ & Appointments assigned to the Customer \\
		\hline
		chat\_messages       & List$<$ChatMessage$>$ & Chat Messages sent by the Customer    \\
		\hline
		quotes               & List$<$Quote$>$       & Quotes initiated by the Customer      \\
		\hline
		vehicles             & List$<$Vehicle$>$     & Vehicles associated with the Customer \\
		\hline
	\end{tabular}
\end{table}

\textbf{Appointment}

\begin{table}[H]
	\begin{tabular}{|p{0.25\textwidth}|p{0.25\textwidth}|p{0.40\textwidth}|}
		\hline
		\textbf{Output Name} & \textbf{Output Type}   & \textbf{Description}           \\
		\hline
		id                   & String                 & ID of Appointment              \\
		\hline
		create\_time         & String                 & Create Time of Appointment     \\
		\hline
		update\_time         & String                 & Update Time of Appointment     \\
		\hline
		quote                & Optional$<$Quote$>$    & Associated Quote               \\
		\hline
		work\_order          & Work Order             & Associated Work Order          \\
		\hline
		vehicle              & Vehicle                & Associated Vehicle             \\
		\hline
		service\_type        & String                 & Type of Service                \\
		\hline
		employee             & Optional$<$Employee$>$ & Assigned Employee              \\
		\hline
		customer             & Customer               & Assigned Customer              \\
		\hline
		status               & String                 & Progress Status of Appointment \\
		\hline
		start\_time          & DateTime               & Start Time                     \\
		\hline
		end\_time            & DateTime               & End Time                       \\
		\hline
		shop                 & Shop                   & Associated Shop                \\
		\hline
	\end{tabular}
\end{table}

\textbf{Quote}

\begin{table}[H]
	\begin{tabular}{|p{0.25\textwidth}|p{0.25\textwidth}|p{0.40\textwidth}|}
		\hline
		\textbf{Output Name} & \textbf{Output Type}      & \textbf{Description}                                    \\
		\hline
		id                   & String                    & ID of Quote                                             \\
		\hline
		create\_time         & String                    & Create Time of Quote                                    \\
		\hline
		update\_time         & String                    & Update Time of Quote                                    \\
		\hline
		customer             & Customer                  & Assigned Customer                                       \\
		\hline
		shop                 & Shop                      & Associated Shop                                         \\
		\hline
		appointment          & Optional$<$Appointment$>$ & Associated Appointment                                  \\
		\hline
		chat\_messages       & List$<$ChatMessage$>$     & Associated Chat Messages                                \\
		\hline
		status               & String                    & Status indicating whether a invitation has been created \\
		\hline
		estimated\_price     & $\mathbb{R}+$             & Estimated price of job                                  \\
		\hline
		duration             & $\mathbb{R}+$             & Estimated time need for job                             \\
		\hline
	\end{tabular}
\end{table}

\textbf{ChatMessage}

\begin{table}[H]
	\begin{tabular}{|p{0.25\textwidth}|p{0.25\textwidth}|p{0.40\textwidth}|}
		\hline
		\textbf{Output Name} & \textbf{Output Type} & \textbf{Description}        \\
		\hline
		id                   & String               & ID of Chat Message          \\
		\hline
		create\_time         & String               & Create Time of Chat Message \\
		\hline
		update\_time         & String               & Update Time of Chat Message \\
		\hline
		message              & String               & Chat Message                \\
		\hline
		quote                & Quote                & Associated Quote            \\
		\hline
		customer             & Customer             & Associated Customer         \\
		\hline
		shop                 & Shop                 & Associated Shop             \\
		\hline
	\end{tabular}
\end{table}

\textbf{Vehicle}

\begin{table}[H]
	\begin{tabular}{|p{0.25\textwidth}|p{0.25\textwidth}|p{0.40\textwidth}|}
		\hline
		\textbf{Output Name} & \textbf{Output Type}  & \textbf{Description}                 \\
		\hline
		id                   & String                & ID of the Vehicle                    \\
		\hline
		create\_time         & String                & Create Time of the Vehicle           \\
		\hline
		update\_time         & String                & Update Time of the Vehicle           \\
		\hline
		year                 & $\mathbb{N}$          & Model Year of the Vehicle            \\
		\hline
		make                 & String                & Make of the Vehicle                  \\
		\hline
		model                & String                & Model of the Vehicle                 \\
		\hline
		vin                  & String                & VIN of the Vehicle                   \\
		\hline
		license\_plate       & String                & License Plate of the Vehicle         \\
		\hline
		customer             & Customer              & Associated Customer                  \\
		\hline
		appointments         & List$<$Appointment$>$ & Appointments assigned to the Vehicle \\
		\hline
	\end{tabular}
\end{table}

\textbf{WorkOrder}

\begin{table}[H]
	\begin{tabular}{|p{0.25\textwidth}|p{0.25\textwidth}|p{0.40\textwidth}|}
		\hline
		\textbf{Output Name} & \textbf{Output Type} & \textbf{Description}      \\
		\hline
		id                   & String               & ID of Work Order          \\
		\hline
		create\_time         & String               & Create Time of Work Order \\
		\hline
		update\_time         & String               & Update Time of Work Order \\
		\hline
		appointment          & Appointment          & Associated Appointment    \\
		\hline
		title                & String               & Title of Work Order       \\
		\hline
		customer             & Customer             & Assigned Customer         \\
		\hline
		vehicle              & Vehicle              & Associated Vehicle        \\
		\hline
		employee             & Employee             & Assigned Employee         \\
		\hline
		body                 & String               & Work Order Details        \\
		\hline
		shop                 & Shop                 & Associated Shop           \\
		\hline
	\end{tabular}
\end{table}

\textbf{Shop}

\begin{table}[H]
	\begin{tabular}{|p{0.25\textwidth}|p{0.25\textwidth}|p{0.40\textwidth}|}
		\hline
		\textbf{Output Name} & \textbf{Output Type}  & \textbf{Description}               \\
		\hline
		id                   & String                & ID of Shop                         \\
		\hline
		create\_time         & String                & Create Time of Shop                \\
		\hline
		update\_time         & String                & Update Time of Shop                \\
		\hline
		name                 & String                & Name of Shop                       \\
		\hline
		address              & String                & Address of Shop                    \\
		\hline
		city                 & String                & City of Shop                       \\
		\hline
		province             & String                & Province of Shop                   \\
		\hline
		postal\_code         & String                & Postal Code of Shop                \\
		\hline
		email                & String                & Email of Shop                      \\
		\hline
		phone\_number        & String                & Phone Number of Shop               \\
		\hline
		hours\_of\_operation & JSON                  & Hours of Operation of Shop         \\
		\hline
		appointments         & List$<$Appointment$>$ & Appointments assigned to Shop      \\
		\hline
		employees            & List$<$Employee$>$    & List of Employee assigned to Shop  \\
		\hline
		chat\_messages       & List$<$ChatMessage$>$ & List of Chat Messages sent by Shop \\
		\hline
		quotes               & List$<$Quote$>$       & List of Quotes                     \\
		\hline
		services             & List$<$Service$>$     & List of offered Services           \\
		\hline
	\end{tabular}
\end{table}

\textbf{Part}

\begin{table}[H]
	\begin{tabular}{|p{0.25\textwidth}|p{0.25\textwidth}|p{0.40\textwidth}|}
		\hline
		\textbf{Output Name} & \textbf{Output Type} & \textbf{Description}                       \\
		\hline
		quantity             & $\mathbb{N}$         & Number of Parts                            \\
		\hline
		cost                 & $\mathbb{R}+$        & Price of Parts                             \\
		\hline
		name                 & String               & Name of Part                               \\
		\hline
		condition            & String               & Condition of Part  (i.e., New or Used)     \\
		\hline
		build                & String               & Origin of Part (i.e., OEM or After Market) \\
		\hline
	\end{tabular}
\end{table}

\textbf{Service}

\begin{table}[H]
	\begin{tabular}{|p{0.25\textwidth}|p{0.25\textwidth}|p{0.40\textwidth}|}
		\hline
		\textbf{Output Name} & \textbf{Output Type}    & \textbf{Description}                \\
		\hline
		id                   & String                  & ID of Service                       \\
		\hline
		create\_time         & String                  & Create Time of Service              \\
		\hline
		update\_time         & String                  & Update Time of Service              \\
		\hline
		name                 & String                  & Name of Service                     \\
		\hline
		description          & String                  & Description of Service              \\
		\hline
		estimated\_time      & $\mathbb{N}$            & Estimated time needed for a Service \\
		\hline
		total\_price         & Optional$<\mathbb{R}+>$ & Estimated price for a Service       \\
		\hline
		parts                & List$<$Part$>$          & List of parts for a Service         \\
		\hline
		shop                 & Shop                    & Shop which offers the Service       \\
		\hline
		type                 & String                  & Type of Service                     \\
		\hline
	\end{tabular}
\end{table}

\subsubsection{Exported Access Programs}

None
\subsection{Semantics}

\subsubsection{State Variables}

None

\subsubsection{Environment Variables}

None

\subsubsection{Assumptions}

None

\subsubsection{Access Routine Semantics}

None

\subsubsection{Local Functions}

None

\newpage
\section{MIS of Shop Module} \label{mShop}
\subsection{Module}

shopService

\subsection{Uses}

Database Driver Module

\subsection{Syntax}

\subsubsection{Exported Constants}

None

\subsubsection{Exported Types}

\textbf{CreateShopType}

\begin{table}[H]
	\begin{tabular}{|p{0.25\textwidth}|p{0.25\textwidth}|p{0.40\textwidth}|}
		\hline
		\textbf{Output Name} & \textbf{Output Type}                & \textbf{Description}           \\
		\hline
		name                 & String                              & Name of Shop                   \\
		\hline
		address              & String                              & Address of shop                \\
		\hline
		city                 & String                              & City of Shop                   \\
		\hline
		province             & String                              & Province of Shop               \\
		\hline
		postal\_code         & String                              & Postal Code of Shop            \\
		\hline
		email                & String                              & Email of Shop                  \\
		\hline
		phone\_number        & String                              & Phone Number of Shop           \\
		\hline
		location             & tuple of (lat, long : $\mathbb{R}$) & Latitude and Longitude of Shop \\
		\hline
	\end{tabular}
\end{table}

\textbf{UpdateShopType}

\begin{table}[H]
	\begin{tabular}{|p{0.25\textwidth}|p{0.25\textwidth}|p{0.40\textwidth}|}
		\hline
		\textbf{Output Name} & \textbf{Output Type} & \textbf{Description}       \\
		\hline
		name                 & Optional$<$String$>$ & Name of Shop               \\
		\hline
		address              & Optional$<$String$>$ & Address of shop            \\
		\hline
		city                 & Optional$<$String$>$ & City of Shop               \\
		\hline
		province             & Optional$<$String$>$ & Province of Shop           \\
		\hline
		postal\_code         & Optional$<$String$>$ & Postal Code of Shop        \\
		\hline
		hours\_of\_operation & Optional$<$JSON$>$   & Hours of Operation of Shop \\
		\hline
		email                & Optional$<$String$>$ & Email of Shop              \\
		\hline
		phone\_number        & Optional$<$String$>$ & Phone Number of Shop       \\
		\hline
	\end{tabular}
\end{table}

\subsubsection{Exported Access Programs}

\begin{center}
	\begin{adjustbox}{width=\textwidth}
		\begin{tabular}{llll}
			\hline
			\textbf{Name}     & \textbf{In}                                        & \textbf{Out}               & \textbf{Exceptions}                         \\
			\hline
			getShopById       & String                                             & Shop $\lor$ None           & ~                                           \\
			updateShopById    & String, UpdateShopType                             & Shop                       & ShopNotFoundException, InvalidTimeException \\
			getShopsByName    & String, Optional$<$String$>$, Optional$<$String$>$ & List$<$Shop$>$ $\lor$ None & ~                                           \\
			getShopsByService & String, Optional$<$String$>$, Optional$<$String$>$ & List$<$Shop$>$ $\lor$ None & ~                                           \\
			\hline
		\end{tabular}
	\end{adjustbox}
\end{center}

\subsection{Semantics}

\subsubsection{State Variables}

\subsubsection{Environment Variables}

User's Display \\ Database

\subsubsection{Assumptions}

None

\subsubsection{Access Routine Semantics}

\noindent getShopById($id$):
\begin{itemize}
	\item output: $out :=$ A Shop such that it contains the ID, $id$, from the shop database table else None.
\end{itemize}

\noindent updateShopById($id, patch$):
\begin{itemize}
	\item transition: Update all fields of a Shop, with an ID equal to $id$, with fields in $patch$ in shop
	      database table. Location is updated locally through the getLatLong(address) local function.
	\item output: $out := getShopById(id) = \text{Shop} \Rightarrow (\text{Shop} = \text{None} $ \\
		      $\Rightarrow \text{ShopNotFoundException } |\ \text{Shop} \neq \text{None}$ \\ $\Rightarrow
		      \text{update all fields of Shop with fields in } patch \text{ in shop database table})$
	\item exception: $exc := getShopById(id) = None \Rightarrow \text{ShopNotFoundException}$ \\ $|\ \exists
		      day \in hours\_of\_operation : day.open\_time > day.close\_time$ \\ $\Rightarrow
		      \text{InvalidTimeException}$
\end{itemize}

\noindent getShopsByName($name, min\_distance, max\_distance$):
\begin{itemize}
	\item output: $out := getShopsByName(name) = (\forall s \in Shops | s.name \in name \land min\_distance <
		      distance(s, user) < max\_distance)$
\end{itemize}

\noindent getShopsByService($service, min\_distance, max\_distance$):
\begin{itemize}
	\item output: $out := getShopsByService(service) = (\forall s \in Shops | \exists serv \in
		      getServicesByShopId(s) | serv \in service \land min\_distance < distance(s, user) < max\_distance)$
\end{itemize}

\subsubsection{Local Functions}

None

% distance: $\mathbb{R}$, $\mathbb{R}$, $\mathbb{R}$, $\mathbb{R}$ $\rightarrow$ $\mathbb{R}$ \\

% distance(shop\_lat, shop\_long, user\_lat, user\_long) $\equiv (3959 \cdot
% 	\arccos(\cos(\text{shop\_lat})) \cdot \cos(\text{user\_lat}) + \cos(\text{shop\_long} -
% 		\text{user\_long}) + \sin(\text{shop\_lat}) \cdot \sin(\text{user\_lat}))$

% getLatLong: String $\rightarrow$ $\mathbb{R}$, $\mathbb{R}$ \\

% getLatLong(address) $\equiv$ the latitude and longitude of the address, as defined by the Google
% Maps API. \\ \newpage

\section{MIS of Users Module} \label{mUsers}
\subsection{Module}

userService

\subsection{Uses}

Database Driver Module

\subsection{Syntax}

\subsubsection{Exported Constants}

None

\subsubsection{Exported Types}

\textbf{CreateCustomerType}

\begin{table}[H]
	\begin{tabular}{|p{0.25\textwidth}|p{0.25\textwidth}|p{0.40\textwidth}|}
		\hline
		\textbf{Output Name}   & \textbf{Output Type} & \textbf{Description}                \\
		\hline
		email                  & String               & Email of Customer                   \\
		\hline
		password               & String               & Password of Customer Account        \\
		\hline
		first\_name            & String               & First Name of Customer              \\
		\hline
		last\_name             & String               & Last Name of Customer               \\
		\hline
		phone\_number          & String               & Phone Number of Customer            \\
		\hline
		vehicle.year           & $\mathbb{N}$         & Model Year of Customer's Vehicle    \\
		\hline
		vehicle.make           & String               & Make of Customer's Vehicle          \\
		\hline
		vehicle.model          & String               & Model of Customer's Vehicle         \\
		\hline
		vehicle.vin            & String               & VIN of Customer's Vehicle           \\
		\hline
		vehicle.license\_plate & String               & License Plate of Customer's Vehicle \\
		\hline
	\end{tabular}
\end{table}

\textbf{CreateEmployeeType}

\begin{table}[H]
	\begin{tabular}{|p{0.25\textwidth}|p{0.25\textwidth}|p{0.40\textwidth}|}
		\hline
		\textbf{Output Name} & \textbf{Output Type} & \textbf{Description}               \\
		\hline
		email                & String               & Email of Employee                  \\
		\hline
		password             & String               & Password of Employee's Account     \\
		\hline
		first\_name          & String               & First Name of Employee             \\
		\hline
		last\_name           & String               & Last Name of Employee              \\
		\hline
		phone\_number        & String               & Phone Number of Employee           \\
		\hline
		shop\_id             & String               & ID of Shop that Employee Works For \\
		\hline
	\end{tabular}
\end{table}

\textbf{CreateShopOwnerType}

\begin{table}[H]
	\begin{tabular}{|p{0.25\textwidth}|p{0.25\textwidth}|p{0.40\textwidth}|}
		\hline
		\textbf{Output Name} & \textbf{Output Type} & \textbf{Description}                     \\
		\hline
		email                & String               & Email of Shop Owner                      \\
		\hline
		password             & String               & Password of Shop Owner's Account         \\
		\hline
		first\_name          & String               & First Name of Shop Owner                 \\
		\hline
		last\_name           & String               & Last Name of Shop Owner                  \\
		\hline
		phone\_number        & String               & Phone Number of Shop Owner               \\
		\hline
		shop.name            & String               & Name of Shop owned by Shop Owner         \\
		\hline
		shop.address         & String               & Address of Shop owned by Shop Owner      \\
		\hline
		shop.postal\_code    & String               & Postal Code of Shop owned by Shop Owner  \\
		\hline
		shop.city            & String               & City of Shop owned by Shop Owner         \\
		\hline
		shop.province        & String               & Province of Shop owned by Shop Owner     \\
		\hline
		shop.email           & String               & Email of Shop owned by Shop Owner        \\
		\hline
		shop.phone\_number   & String               & Phone Number of Shop owned by Shop Owner \\
		\hline
	\end{tabular}
\end{table}

\textbf{AuthorizeReturnType}

\begin{table}[H]
	\begin{tabular}{|p{0.25\textwidth}|p{0.25\textwidth}|p{0.40\textwidth}|}
		\hline
		\textbf{Output Name} & \textbf{Output Type} & \textbf{Description} \\
		\hline
		id                   & String               & User ID              \\
		\hline
		firstName            & String               & First Name of User   \\
		\hline
		lastName             & String               & Last Name of User    \\
		\hline
		email                & String               & Email of User        \\
		\hline
		type                 & String               & Type of User         \\
		\hline
	\end{tabular}
\end{table}

\subsubsection{Exported Access Programs}

\begin{center}
	\begin{adjustbox}{width=\textwidth}
		\begin{tabular}{llll}
			\hline
			\textbf{Name}   & \textbf{In}         & \textbf{Out}                         & \textbf{Exceptions}             \\
			\hline
			createCustomer  & CreateCustomerType  & Customer                             & CustomerAlreadyExistsException  \\
			createEmployee  & CreateEmployeeType  & Employee                             & EmployeeAlreadyExistsException  \\
			createShopOwner & CreateShopOwnerType & Employee                             & ShopOwnerAlreadyExistsException \\
			getUserByEmail  & String              & Customer $\lor$ Employee $\lor$ None & ~                               \\
			authorize       & String, String      & AuthorizeReturnType                  & UnauthorizeException            \\
			\hline
		\end{tabular}
	\end{adjustbox}
\end{center}

\subsection{Semantics}

\subsubsection{State Variables}

None

\subsubsection{Environment Variables}

User's Display \\ Database

\subsubsection{Assumptions}

None

\subsubsection{Access Routine Semantics}

\noindent createCustomer($customer$):
\begin{itemize}
	\item transition: new Customer(customer), save customer to customer database table
	\item output: $out := (exc = \text{CustomerAlreadyExistsException}$ \\ $\Rightarrow ``\text{User with
			      email address already exists.}" |\ \neg exc \Rightarrow \text{new Customer(\emph{customer})})$
	\item exception: $exc := getUserByEmail(customer.email) \neq \text{None}$ \\ $\Rightarrow
		      \text{CustomerAlreadyExistsException}$
\end{itemize}

\noindent createEmployee($employee$):
\begin{itemize}
	\item transition: new Employee(employee), save employee to employee database table
	\item output: $out := (exc = \text{EmployeeAlreadyExistsException}$ \\ $\Rightarrow ``\text{User with
			      email address already exists.}" |\ \neg exc \Rightarrow \text{new Employee(\emph{employee})})$
	\item exception: $exc := getUserByEmail(employee.email) \neq \text{None}$ \\ $\Rightarrow
		      \text{EmployeeAlreadyExistsException}$
\end{itemize}

\noindent createShopOwner($shopOwner$):
\begin{itemize}
	\item transition: new Employee(shopOwner), save shop owner to employee database table and creates a new
	      shop
	\item output: $out := (exc = \text{ShopOwnerAlreadyExistsException}$ \\ $\Rightarrow ``\text{User with
			      email address already exists.}" |\ \neg exc \Rightarrow \text{new Employee(\emph{shopOwner})} \land
		      \text{new Shop(\emph{shop.name, shop.address, shop.city, shop.province, shop.postal\_code, \\
				      shop.email, shop.phone\_number})})$
	\item exception: $exc := getUserByEmail(shopOwner.email) \neq \text{None}$ \\ $\Rightarrow
		      \text{ShopOwnerAlreadyExistsException}$
\end{itemize}

\noindent getUserByEmail($e$):
\begin{itemize}
	\item output: $out :=$ A User such that it contains the email, $e$, from the customer database table or
	      employees database table, else None.
\end{itemize}

\noindent authorize($email, password$):
\begin{itemize}
	\item output: $out := getUserByEmail(email) = \text{User} \Rightarrow $ \\ $(\text{User} = \text{None}
		      \Rightarrow exc = \text{UserNotFoundException} \Rightarrow \text{``User not found."}$ \\ $|\
		      \text{User.password} = password \Rightarrow \text{AuthorizeReturnType}$ \\ $|\ \text{User.password}
		      \neq password \Rightarrow exc = \text{UnauthorizeException} \Rightarrow \text{``Unauthorized"})$
\end{itemize}

\subsubsection{Local Functions}

\noindent Shop($name, address, city, province, postal\_code, email, phone\_number$):
\begin{itemize}
	\item transition: Creates a new shop object with the following parameters: the provided name, address,
	      city, province, postal code, email, and phone number.
\end{itemize}

\newpage

\section{MIS of Employee Management Module} \label{mEmployeeManagement}
\subsection{Module}

employeeManagementService

\subsection{Uses}

Database Driver Module

\subsection{Syntax}

\subsubsection{Exported Constants}

None

\subsubsection{Exported Types}

\textbf{UpdateEmployeeType}

\begin{table}[H]
	\begin{tabular}{|p{0.25\textwidth}|p{0.25\textwidth}|p{0.40\textwidth}|}
		\hline
		\textbf{Output Name} & \textbf{Output Type} & \textbf{Description}     \\
		\hline
		email                & Optional$<$String$>$ & Email of Employee        \\
		\hline
		first\_name          & Optional$<$String$>$ & First Name of Employee   \\
		\hline
		last\_name           & Optional$<$String$>$ & Last Name of Employee    \\
		\hline
		phone\_number        & Optional$<$String$>$ & Phone Number of Employee \\
		\hline
	\end{tabular}
\end{table}

\subsubsection{Exported Access Programs}

\begin{center}
	\begin{adjustbox}{width=\textwidth}
		\begin{tabular}{llll}
			\hline
			\textbf{Name}        & \textbf{In}                & \textbf{Out}       & \textbf{Exceptions}   \\
			\hline
			getEmployeeById      & String                     & Employee           & ~                     \\
			getEmployeesByShopId & String                     & List$<$Employee$>$ & ~                     \\
			updateEmployeeById   & String, UpdateEmployeeType & ~                  & UserNotFoundException \\
			suspendEmployeeById  & String                     & ~                  & ~                     \\
			\hline
		\end{tabular}
	\end{adjustbox}
\end{center}

\subsection{Semantics}

\subsubsection{State Variables}

None

\subsubsection{Environment Variables}

User's Display \\ Database

\subsubsection{Assumptions}

None

\subsubsection{Access Routine Semantics}

\noindent getEmployeeById($employeeId$):
\begin{itemize}
	\item output: $out :=$ An employee such that it's ID matches $employeeId$ from the employee database
	      table else None.
\end{itemize}

\noindent getEmployeesByShopId($shopId$):
\begin{itemize}
	\item output: $out :=$ List of employees from the shop database table associated with the $shopId$, else
	      None.
\end{itemize}

\noindent updateEmployeeById($employeeId, patch$):
\begin{itemize}
	\item transition: Update all fields of Employee with fields in $patch$ in employee database table
	\item output: $out := getEmployeeById(id) = \text{Employee} \Rightarrow (\text{Employee} = \text{None} $
		      \\ $\Rightarrow \text{UserNotFoundException } |\ \text{Employee} \neq \text{None}$ \\ $\Rightarrow
		      \text{update all fields of Employee with fields in } patch \text{ in service database table})$
	\item exception: $exc := getEmployeeById(id) = None \Rightarrow \text{UserNotFoundException}$
\end{itemize}

\noindent suspendEmployeeById($employeeId$):
\begin{itemize}
	\item transition: Change the state of the employee to ``suspended" where their ID matches $employeeId$ in
	      the employee database table.
\end{itemize}

\subsubsection{Local Functions}

None

\newpage

\section{MIS of Services Module} \label{mServices}
\subsection{Module}

serviceService

\subsection{Uses}

Database Driver Module

\subsection{Syntax}

\subsubsection{Exported Constants}

None

\subsubsection{Exported Types}

\textbf{CreateServiceType}

\begin{table}[H]
	\begin{tabular}{|p{0.25\textwidth}|p{0.25\textwidth}|p{0.40\textwidth}|}
		\hline
		\textbf{Output Name} & \textbf{Output Type}    & \textbf{Description}                     \\
		\hline
		name                 & String                  & Name of Service                          \\
		\hline
		description          & Optional$<$String$>$    & Description of Service                   \\
		\hline
		estimated\_time      & Optional$<$String$>$    & Estimated Time of Service                \\
		\hline
		parts                & List$<$Part$>$          & Estimated Parts needed for Service       \\
		\hline
		total\_price         & Optional$<\mathbb{R}+>$ & Estimated Price needed for Service       \\
		\hline
		shop\_id             & String                  & Shop that Service is available at        \\
		\hline
		type                 & String                  & Type of Service (e.g., Canned or Custom) \\
		\hline
	\end{tabular}
\end{table}

\textbf{UpdateServiceType}

\begin{table}[H]
	\begin{tabular}{|p{0.25\textwidth}|p{0.25\textwidth}|p{0.40\textwidth}|}
		\hline
		\textbf{Output Name} & \textbf{Output Type}       & \textbf{Description}                     \\
		\hline
		name                 & Optional$<$String$>$       & Name of Service                          \\
		\hline
		description          & Optional$<$String$>$       & Description of Service                   \\
		\hline
		estimated\_time      & Optional$<$String$>$       & Estimated Time of Service                \\
		\hline
		parts                & Optional$<$List$<$Part$>>$ & Estimated Parts needed for Service       \\
		\hline
		total\_price         & Optional$<\mathbb{R}+>$    & Estimated Price needed for Service       \\
		\hline
		type                 & Optional$<$String$>$       & Type of Service (e.g., Canned or Custom) \\
		\hline
	\end{tabular}
\end{table}

\subsubsection{Exported Access Programs}

\begin{center}
	\begin{adjustbox}{width=\textwidth}
		\begin{tabular}{llll}
			\hline
			\textbf{Name}             & \textbf{In}               & \textbf{Out}        & \textbf{Exceptions}      \\
			\hline
			createService             & CreateServiceType         & Service             & ~                        \\
			getServiceById            & String                    & Service $\lor$ None & ~                        \\
			getServicesByShopId       & String                    & List$<$Service$>$   & ~                        \\
			getCannedServicesByShopId & String                    & List$<$Service$>$   & ~                        \\
			getCustomServicesByShopId & String                    & List$<$Service$>$   & ~                        \\
			updateServiceById         & String, UpdateServiceType & Service             & ServiceNotFoundException \\
			deleteServiceById         & String                    & ~                   & ~                        \\
			\hline
		\end{tabular}
	\end{adjustbox}
\end{center}

\subsection{Semantics}

\subsubsection{State Variables}

None

\subsubsection{Environment Variables}

User's Display \\ Database

\subsubsection{Assumptions}

None

\subsubsection{Access Routine Semantics}

\noindent createService($service$):
\begin{itemize}
	\item transition: new Service(service), save service to service database table
	\item output: $out := \text{new Service(\emph{service})}$
\end{itemize}

\noindent getServiceById($id$):
\begin{itemize}
	\item output: $out :=$ A Service such that it's ID matches $id$ from the service database table else
	      None.
\end{itemize}

\noindent getServicesByShopId($shopId$):
\begin{itemize}
	\item output: $out :=$ A list of Service such that it's shop ID matches $shopId$ from the service
	      database table else an empty list.
\end{itemize}

\noindent getCannedServicesByShopId($shopId$):
\begin{itemize}
	\item output: $out :=$ A list of Service such that it's shop ID matches $shopId$ and it's type is
	      ``Canned" from the service database table else an empty list.
\end{itemize}

\noindent getCustomServicesByShopId($shopId$):
\begin{itemize}
	\item output: $out :=$ A list of Service such that it's shop ID matches $shopId$ and it's type is
	      ``Custom" from the service database table else an empty list.
\end{itemize}

\noindent updateServiceById($id, patch$):
\begin{itemize}
	\item transition: Update all fields of Service with fields in $patch$ in service database table
	\item output: $out := getServiceById(id) = \text{Service} \Rightarrow (\text{Service} = \text{None} $ \\
		      $\Rightarrow \text{ServiceNotFoundException } |\ \text{Service} \neq \text{None}$ \\ $\Rightarrow
		      \text{update all fields of Service with fields in } patch \text{ in service database table})$
	\item exception: $exc := getServiceById(id) = None \Rightarrow \text{ServiceNotFoundException}$
\end{itemize}

\noindent deleteServiceById($id$):
\begin{itemize}
	\item transition: Delete all Services where their ID matches $id$ from the service database table.
\end{itemize}

\subsubsection{Local Functions}

None

\newpage

\section{MIS of Appointments Module} \label{mAppointments}
\subsection{Module}

appointmentService

\subsection{Uses}

Database Driver Module

\subsection{Syntax}

\subsubsection{Exported Constants}

None

\subsubsection{Exported Types}

\textbf{CreateAppointmentType}

\begin{table}[H]
	\begin{tabular}{|l|l|l|}
		\hline
		\textbf{Output Name} & \textbf{Output Type} & \textbf{Description}                    \\
		\hline
		quote\_id            & Optional$<$String$>$ & Associated Quote                        \\
		\hline
		service\_type        & String               & Type of Service                         \\
		\hline
		price                & $\mathbb{Q}$         & Estimated Price                         \\
		\hline
		employee\_id         & Optional$<$String$>$ & Assigned Employee                       \\
		\hline
		start\_time          & Date                 & Start Time of Appointment               \\
		\hline
		end\_time            & Date                 & End Time of Appointment                 \\
		\hline
		vehicle\_id          & String               & Vehicle Being Serviced                  \\
		\hline
		customer\_id         & String               & Customer the Appointment is for         \\
		\hline
		shop\_id             & String               & Shop the Appointment is taking place at \\
		\hline
	\end{tabular}
\end{table}

\textbf{UpdateAppointmentType}

\begin{table}[H]
	\begin{tabular}{|l|l|l|}
		\hline
		\textbf{Output Name} & \textbf{Output Type}   & \textbf{Description}             \\
		\hline
		quote\_id            & Optional$<$String$>$   & Associated Quote                 \\
		\hline
		work\_order\_id      & Optional$<$String$>$   & Associated Work Order            \\
		\hline
		service\_type        & Optional$<$String$>$   & Type of Service                  \\
		\hline
		price                & Optional$<\mathbb{Q}>$ & Estimated Price                  \\
		\hline
		employee\_id         & Optional$<$String$>$   & Assigned Employee                \\
		\hline
		start\_time          & Optional$<$Date$>$     & Start Time of Appointment        \\
		\hline
		end\_time            & Optional$<$Date$>$     & End Time of Appointment          \\
		\hline
		vehicle\_id          & Optional$<$String$>$   & Vehicle Being Serviced           \\
		\hline
		status               & Optional$<$String$>$   & Acceptance status of appointment \\
		\hline
	\end{tabular}
\end{table}

\subsubsection{Exported Access Programs}

\begin{center}
	\begin{adjustbox}{width=\textwidth}
		\begin{tabular}{llll}
			\hline
			\textbf{Name}               & \textbf{In}                   & \textbf{Out}            & \textbf{Exceptions}                                \\
			\hline
			createAppointment           & CreateAppointmentType         & Appointment             & InvalidTimeException, MissingDataException         \\
			getAppointmentById          & String                        & Appointment $\lor$ None & ~                                                  \\
			getAppointmentsByShopId     & String                        & List$<$Appointment$>$   & ~                                                  \\
			getAppointmentsByCustomerId & String                        & List$<$Appointment$>$   & ~                                                  \\
			updateAppointmentById       & String, UpdateAppointmentType & Appointment             & AppointmentNotFoundException, InvalidTimeException \\
			deleteAppointment           & String                        & ~                       & ~                                                  \\
			\hline
		\end{tabular}
	\end{adjustbox}
\end{center}

\subsection{Semantics}

\subsubsection{State Variables}

None

\subsubsection{Environment Variables}

User's Display \\ Database

\subsubsection{Assumptions}

None

\subsubsection{Access Routine Semantics}

\noindent createAppointment($appointment$):
\begin{itemize}
	\item transition: new Appointment(appointment) as ``Pending Approval", save appointment to appointment
	      database table. new WorkOrder() connected to the newly created appointment, save work order to work
	      order database table.
	\item output: $out := (exc = \text{InvalidTimeException}$ \\ $\Rightarrow \text{``Invalid start time
			      and/or end time."} |\ \neg exc \Rightarrow \text{new Appointment(\emph{appointment})})$
	\item exception: $exc := (appointment.vehicle\_id = \text{None} \lor appointment.customer\_id =
		      \text{None} \lor appointment.shop\_id = \text{None} \Rightarrow \text{MissingDataException})$ \\
	      $|\ (\neg isAppointmentValid(appointment.shop\_id, appointment.start\_time, appointment.end\_time)
	      $ \\ $\Rightarrow \text{InvalidTimeException})$
\end{itemize}

\noindent getAppointmentById($id$):
\begin{itemize}
	\item output: $out :=$ An Appointment such that it's ID matches $id$ from the appointment database table
	      else None.
\end{itemize}

\noindent getAppointmentsByShopId($shopId$):
\begin{itemize}
	\item output: $out :=$ A list of Appointments such that their shop ID matches $shopId$ from the
	      appointment database table else an empty list.
\end{itemize}

\noindent getAppointmentsByCustomerId($customerId$):
\begin{itemize}
	\item output: $out :=$ A list of Appointments such that their customer ID matches $customerId$ from the
	      appointment database table else an empty list.
\end{itemize}

\noindent updateAppointmentById($id, patch$):
\begin{itemize}
	\item transition: Update all fields of an Appointment, with an ID equal to $id$, with fields in $patch$
	      in appointment database table
	\item output: $out := getAppointmentById(id) = \text{Appointment}$ \\ $\Rightarrow (\text{Appointment} =
		      \text{None} \Rightarrow \text{AppointmentNotFoundException }$ \\ $|\ \text{Appointment} \neq
		      \text{None}$ \\ $\Rightarrow \text{update all fields of Appointment with fields in } patch \text{
			      in appointment database table})$
	\item exception: $exc := (getAppointmentById(id) = None \Rightarrow \text{AppointmentNotFoundException})$
	      \\ $|\ (getAppointmentById(id) = Appointment \Rightarrow (patch.start\_time \neq \text{None } \land
		      patch.end\_time \neq \text{None } \land \neg isAppointmentValid(Appointment.shop\_id,
		      patch.start\_time, patch.end\_time)) \lor (patch.start\_time \neq \text{None}$ \\ $\land \neg
		      isAppointmentValid(Appointment.shop\_id, patch.start\_time, Appointment.end\_time)) \lor
		      (patch.end\_time \neq \text{None}$ \\ $\land \neg isAppointmentValid(Appointment.shop\_id,
		      Appointment.start\_time, patch.end\_time))$ \\ $\Rightarrow \text{InvalidTimeException})$
\end{itemize}

\noindent deleteAppointment($id$):
\begin{itemize}
	\item transition: Delete all Appointments where their ID matches $id$ from the appointment database
	      table.
\end{itemize}

\subsubsection{Local Functions}

\noindent acceptAppointment($appointment$):
\begin{itemize}
	\item transition: Mark the appointment as ``Accepted". If this appointment will consume the final
	      employee or final work stall available during its specified time slot, mark all other appointment
	      with overlapping time slot as ``Rejected".
\end{itemize}

\noindent isAppointmentValid($shop\_id, start\_time, end\_time$):
\begin{itemize}
	\item output: $out :=$ True if the given appointment is within the hours of operation for the shop else
	      False.
\end{itemize}

\newpage

\section{MIS of Work Orders Module} \label{mWorkOrders}
\subsection{Module}

workOrderService

\subsection{Uses}

Database Driver Module

\subsection{Syntax}

\subsubsection{Exported Constants}

None

\subsubsection{Exported Types}

\textbf{UpdateWorkOrderType}

\begin{table}[H]
	\begin{tabular}{|p{0.25\textwidth}|p{0.25\textwidth}|p{0.40\textwidth}|}
		\hline
		\textbf{Output Name} & \textbf{Output Type} & \textbf{Description}    \\
		\hline
		title                & Optional$<$String$>$ & Title of Work Order     \\
		\hline
		body                 & Optional$<$String$>$ & Work Order Details      \\
		\hline
		employee\_id         & Optional$<$String$>$ & ID of Assigned Employee \\
		\hline
	\end{tabular}
\end{table}

\subsubsection{Exported Access Programs}

\begin{center}
	\begin{adjustbox}{width=\textwidth}
		\begin{tabular}{llll}
			\hline
			\textbf{Name}         & \textbf{In}                 & \textbf{Out}                    & \textbf{Exceptions}        \\
			\hline
			getWorkOrderById      & String                      & WorkOrder $\lor$ None           & ~                          \\
			getWorkOrdersByShopId & String                      & List$<$WorkOrder$>$ $\lor$ None & ~                          \\
			updateWorkOrderById   & String, UpdateWorkOrderType & WorkOrder                       & WorkOrderNotFoundException \\
			\hline
		\end{tabular}
	\end{adjustbox}
\end{center}

\subsection{Semantics}

\subsubsection{State Variables}

None

\subsubsection{Environment Variables}

User's Display \\ Database

\subsubsection{Assumptions}

None

\subsubsection{Access Routine Semantics}
\noindent getWorkOrderById($id$):
\begin{itemize}
	\item output: $out :=$ A Work Order such that it's ID matches $id$ from the work order database table
	      else None.
\end{itemize}

\noindent getWorkOrdersByShopId($shopId$):
\begin{itemize}
	\item output: $out :=$ A list of Work Orders such that their shop ID matches $shopId$ from the work order
	      database table else an empty list.
\end{itemize}

\noindent updateWorkOrderById($id, patch$):
\begin{itemize}
	\item transition: Update all fields of a Work Order, with an ID equal to $id$, with fields in $patch$ in
	      work order database table
	\item output: $out := getWorkOrderById(id) = \text{WorkOrder} \Rightarrow (\text{WorkOrder} = \text{None}
		      $ \\ $\Rightarrow \text{WorkOrderNotFoundException } |\ \text{WorkOrder} \neq \text{None}$ \\
		      $\Rightarrow \text{update all fields of WorkOrder with fields in } patch \text{ in work order
			      database table})$
	\item exception: $exc := getWorkOrderById(id) = None \Rightarrow \text{WorkOrderNotFoundException}$
\end{itemize}

\subsubsection{Local Functions}

None

\newpage

\section{MIS of Quotes Module} \label{mQuotes}
\subsection{Module}

quoteService \\ chatService

\subsection{Uses}

Database Driver Module

\subsection{Syntax}

\subsubsection{Exported Constants}

None

\subsubsection{Exported Types}

\textbf{CreateQuoteType}

\begin{table}[H]
	\begin{tabular}{|p{0.25\textwidth}|p{0.25\textwidth}|p{0.40\textwidth}|}
		\hline
		\textbf{Output Name} & \textbf{Output Type} & \textbf{Description} \\
		\hline
		customer\_id         & String               & ID of Customer       \\
		\hline
		shop\_id             & String               & ID of Shop           \\
		\hline
		service\_id          & String               & ID of Service        \\
		\hline
	\end{tabular}
\end{table}

\textbf{UpdateQuoteType}

\begin{table}[H]
	\begin{tabular}{|p{0.25\textwidth}|p{0.25\textwidth}|p{0.40\textwidth}|}
		\hline
		\textbf{Output Name} & \textbf{Output Type}    & \textbf{Description}                                     \\
		\hline
		status               & Optional$<$String$>$    & Status indicating whether an invitation has been created \\
		\hline
		estimated\_price     & Optional$<\mathbb{R}+>$ & Estimated price of job                                   \\
		\hline
		duration             & Optional$<\mathbb{R}+>$ & Estimated time need for job                              \\
		\hline
		description          & Optional$<$String$>$    & Description of quote                                     \\
		\hline
	\end{tabular}
\end{table}

\textbf{CreateChatMessageType}

\begin{table}[H]
	\begin{tabular}{|p{0.25\textwidth}|p{0.25\textwidth}|p{0.40\textwidth}|}
		\hline
		\textbf{Output Name} & \textbf{Output Type} & \textbf{Description} \\
		\hline
		customer\_id         & Optional$<$String$>$ & ID of Customer       \\
		\hline
		shop\_id             & Optional$<$String$>$ & ID of Shop           \\
		\hline
		message              & String               & Chat message         \\
		\hline
	\end{tabular}
\end{table}

\subsubsection{Exported Access Programs}

\begin{center}
	\begin{adjustbox}{width=\textwidth}
		\begin{tabular}{llll}
			\hline
			\textbf{Name}            & \textbf{In}                   & \textbf{Out}          & \textbf{Exceptions}                            \\
			\hline
			createQuote              & CreateQuoteType               & Quote                 & ~                                              \\
			updateQuoteById          & String, UpdateQuoteType       & Quote                 & QuoteNotFoundException                         \\
			getQuoteById             & String                        & Quote $\lor$ None     & ~                                              \\
			getQuotesByCustomerId    & String                        & List$<$Quote$>$       & ~                                              \\
			getQuotesByShopId        & String                        & List$<$Quote$>$       & ~                                              \\
			deleteQuoteAndChatById   & String                        & ~                     & ~                                              \\
			createChatMessage        & CreateChatMessageType, String & ChatMessage           & MissingSenderException, QuoteNotFoundException \\
			getChatMessagesByQuoteId & String                        & List$<$ChatMessage$>$ & ~                                              \\
			\hline
		\end{tabular}
	\end{adjustbox}
\end{center}

\subsection{Semantics}

\subsubsection{State Variables}

None

\subsubsection{Environment Variables}

User's Display \\ Database

\subsubsection{Assumptions}

None

\subsubsection{Access Routine Semantics}

\noindent createQuote($quote$):
\begin{itemize}
	\item transition: new Quote(quote), save quote to quote database table
	\item output: $out := \text{new Quote(\emph{quote})}$
\end{itemize}

\noindent updateQuoteById($id, patch$):
\begin{itemize}
	\item transition: Update all fields of a Quote, with an ID equal to $id$, with fields in $patch$ in quote
	      database table
	\item output: $out := getQuoteById(id) = \text{Quote} \Rightarrow (\text{Quote} = \text{None} $ \\
		      $\Rightarrow \text{QuoteNotFoundException } |\ \text{Quote} \neq \text{None}$ \\ $\Rightarrow
		      \text{update all fields of Quote with fields in } patch \text{ in quote database table})$
	\item exception: $exc := getQuoteById(id) = None \Rightarrow \text{QuoteNotFoundException}$
\end{itemize}

\noindent getQuoteById($id$):
\begin{itemize}
	\item output: $out :=$ A Quote such that it's ID matches $id$ from the quote database table else None.
\end{itemize}

\noindent getQuotesByCustomerId($customerId$):
\begin{itemize}
	\item output: $out :=$ A list of Quotes such that it's customer ID matches $customerId$ from the quote
	      database table else an empty list.
\end{itemize}

\noindent getQuotesByShopId($shopId$):
\begin{itemize}
	\item output: $out :=$ A list of Quotes such that it's shop ID matches $shopId$ from the quote database
	      table else an empty list.
\end{itemize}

\noindent deleteQuoteAndChatById($id$):
\begin{itemize}
	\item transition: Delete all Quote and ChatMessage where their quote ID matches $id$ from the quote
	      database table and the chat message database table, respectively.
\end{itemize}

\noindent createChatMessage($chatMessage, quoteId$):
\begin{itemize}
	\item transition: new ChatMessage($chatMessage, quoteId$), save quote to quote database table
	\item output: $out := (exc = QuoteNotFoundException \Rightarrow \text{``Quote Not Found" })$ \\ $|\ exc =
		      MissingSenderException \Rightarrow \text{``Missing Sender Information" }$ \\ $|\ \neg exc
		      \Rightarrow \text{new ChatMessage(\emph{chatMessage, quoteId})}$
	\item exception: $exc := (getQuoteById(quoteId) = None \Rightarrow QuoteNotFoundException$ \\ $ |\
		      chatMessage.customer\_id = \text{None} \land chatMessage.shop\_id = \text{None}$ \\ $\Rightarrow
		      MissingSenderException)$
\end{itemize}

\noindent getQuoteById($quoteId$):
\begin{itemize}
	\item output: $out :=$ A list of Chat Messages such that their quote ID matches $quoteId$ from the chat
	      message database table else an empty list.
\end{itemize}

\subsubsection{Local Functions}

None

\newpage

\bibliographystyle{plainnat}
\bibliography{../../../refs/References}

\newpage

\section{Appendix}

\end{document}
