\documentclass[12pt, titlepage]{article}

\usepackage{amsmath, mathtools}
\usepackage[round]{natbib}
\usepackage{amsfonts}
\usepackage{amssymb}
\usepackage{graphicx}
\usepackage{colortbl}
\usepackage{xr}
\usepackage{hyperref}
\usepackage{longtable}
\usepackage{xfrac}
\usepackage{tabularx}
\usepackage{float}
\usepackage{siunitx}
\usepackage{booktabs}
\usepackage{multirow}
\usepackage[section]{placeins}
\usepackage{caption}
\usepackage{fullpage}
\usepackage{array}
\usepackage[nottoc,numbib]{tocbibind}
\usepackage{adjustbox}


\hypersetup{
	bookmarks=true,         % show bookmarks bar?
	colorlinks=true,        % false: boxed links; true: colored links
	linkcolor=red,          % color of internal links (change box color with linkbordercolor)
	citecolor=blue,         % color of links to bibliography
	filecolor=magenta,      % color of file links
	urlcolor=cyan           % color of external links
}


\externaldocument{../../SRS/SRS}

%% Comments

\usepackage{color}

\newif\ifcomments\commentstrue %displays comments
%\newif\ifcomments\commentsfalse %so that comments do not display

\ifcomments
\newcommand{\authornote}[3]{\textcolor{#1}{[#3 ---#2]}}
\newcommand{\todo}[1]{\textcolor{red}{[TODO: #1]}}
\else
\newcommand{\authornote}[3]{}
\newcommand{\todo}[1]{}
\fi

\newcommand{\wss}[1]{\authornote{blue}{SS}{#1}} 
\newcommand{\plt}[1]{\authornote{magenta}{TPLT}{#1}} %For explanation of the template
\newcommand{\an}[1]{\authornote{cyan}{Author}{#1}}

%% Common Parts

\newcommand{\progname}{Sayyara}
\newcommand{\authname}{Team 3, Tiny Coders
	\\ Arkin Modi
	\\ Joy Xiao
	\\ Leon So
	\\ Timothy Choy} % AUTHOR NAMES

\usepackage{hyperref}
\hypersetup{colorlinks=true, linkcolor=blue, citecolor=blue, filecolor=blue,
	urlcolor=blue, unicode=false}
\urlstyle{same}

\usepackage{parskip}
\usepackage{geometry}
\geometry{a4paper, portrait, margin=1in}


\begin{document}

\title{Module Interface Specification for \progname{}}

\author{\authname}

\date{\today}

\maketitle

\pagenumbering{roman}

\section{Revision History}

\begin{table}[hp]
	\caption{Revision History} \label{TblRevisionHistory}
	\begin{tabularx}{\textwidth}{llX}
		\toprule
		\textbf{Date}     & \textbf{Developer(s)} & \textbf{Change}           \\
		\midrule
		December 28, 2022 & Arkin Modi            & Create Revision History   \\
		January 10, 2023  & Arkin Modi            & Create MIS of User Module \\
		\bottomrule
	\end{tabularx}
\end{table}

~\newpage

\section{Symbols, Abbreviations and Acronyms}

See SRS Documentation at \wss{give url}

\wss{Also add any additional symbols, abbreviations or acronyms}

\newpage

\tableofcontents

\newpage

\listoftables

\listoffigures

\newpage

\pagenumbering{arabic}

\section{Introduction}

The following document details the Module Interface Specifications for \wss{Fill in your project
	name and description}

Complementary documents include the System Requirement Specifications and Module Guide. The full
documentation and implementation can be found at \url{...}. \wss{provide the url for your repo}

\section{Notation}

\wss{You should describe your notation.  You can use what is below as
	a starting point.}

The structure of the MIS for modules comes from \citet{HoffmanAndStrooper1995}, with the addition
that template modules have been adapted from \cite{GhezziEtAl2003}. The mathematical notation comes
from Chapter 3 of \citet{HoffmanAndStrooper1995}. For instance, the symbol := is used for a
multiple assignment statement and conditional rules follow the form $(c_1 \Rightarrow r_1 | c_2
	\Rightarrow r_2 | ... | c_n \Rightarrow r_n )$.

The following table summarizes the primitive data types used by \progname.

\begin{center}
	\renewcommand{\arraystretch}{1.2}
	\noindent
	\begin{tabular}{l l p{7.5cm}}
		\toprule
		\textbf{Data Type} & \textbf{Notation} & \textbf{Description}                                             \\
		\midrule
		character          & char              & a single symbol or digit                                         \\
		integer            & $\mathbb{Z}$      & a number without a fractional component in (-$\infty$, $\infty$) \\
		natural number     & $\mathbb{N}$      & a number without a fractional component in [1, $\infty$)         \\
		real               & $\mathbb{R}$      & any number in (-$\infty$, $\infty$)                              \\
		\bottomrule
	\end{tabular}
\end{center}

\noindent
The specification of \progname \ uses some derived data types: sequences, strings, and
tuples. Sequences are lists filled with elements of the same data type. Strings
are sequences of characters. Tuples contain a list of values, potentially of
different types. In addition, \progname \ uses functions, which
are defined by the data types of their inputs and outputs. Local functions are
described by giving their type signature followed by their specification.

\section{Module Decomposition}

The following table is taken directly from the Module Guide document for this project.

\begin{table}[h!]
	\centering
	\begin{tabular}{p{0.3\textwidth} p{0.6\textwidth}}
		\toprule
		\textbf{Level 1}                               & \textbf{Level 2}                \\
		\midrule

		{Hardware-Hiding}                              & ~                               \\
		\midrule

		\multirow{7}{0.3\textwidth}{Behaviour-Hiding}  & Input Parameters                \\
		                                               & Output Format                   \\
		                                               & Output Verification             \\
		                                               & Temperature ODEs                \\
		                                               & Energy Equations                \\
		                                               & Control Module                  \\
		                                               & Specification Parameters Module \\
		\midrule

		\multirow{3}{0.3\textwidth}{Software Decision} & {Sequence Data Structure}       \\
		                                               & ODE Solver                      \\
		                                               & Plotting                        \\
		\bottomrule

	\end{tabular}
	\caption{Module Hierarchy}
	\label{TblMH}
\end{table}

\section{MIS of Users Module} \label{mUsers}
\subsection{Module}

userService

\subsection{Uses}

Database Driver Module

\subsection{Syntax}

\subsubsection{Exported Constants}

None

% \subsubsection{Exported Types}

% \textbf{CreateCustomerType}

% \textbf{CreateEmployeeType}

% \textbf{CreateShopOwnerType}

% \textbf{AuthorizeReturnType}

\subsubsection{Exported Access Programs}

\begin{center}
	\begin{adjustbox}{width=\textwidth}
		\begin{tabular}{llll}
			\hline
			\textbf{Name}   & \textbf{In}         & \textbf{Out}                         & \textbf{Exceptions}             \\
			\hline
			createCustomer  & CreateCustomerType  & Customer                             & CustomerAlreadyExistsException  \\
			createEmployee  & CreateEmployeeType  & Employee                             & EmployeeAlreadyExistsException  \\
			createShopOwner & CreateShopOwnerType & Employee                             & ShopOwnerAlreadyExistsException \\
			getUserByEmail  & String              & Customer $\lor$ Employee $\lor$ None & ~                               \\
			authorize       & String, String      & AuthorizeReturnType                  & ~                               \\
			\hline
		\end{tabular}
	\end{adjustbox}
\end{center}

\subsection{Semantics}

\subsubsection{State Variables}

None

\subsubsection{Environment Variables}

None

\subsubsection{Assumptions}

None

\subsubsection{Access Routine Semantics}

\noindent createCustomer($customer$):
\begin{itemize}
	\item transition: new Customer(customer), save customer to customer database
	\item output: $out := (exc = \text{CustomerAlreadyExistsException}$ \\ $\Rightarrow ``\text{User with
			      email address already exists.}" |\ \neg exc \Rightarrow \text{new Customer(\emph{customer})})$
	\item exception: $exc := getUserByEmail(customer.email) \neq \text{None}$ \\ $\Rightarrow
		      \text{CustomerAlreadyExistsException}$
\end{itemize}

\noindent createEmployee($employee$):
\begin{itemize}
	\item transition: new Employee(employee), save employee to employee database
	\item output: $out := (exc = \text{EmployeeAlreadyExistsException}$ \\ $\Rightarrow ``\text{User with
			      email address already exists.}" |\ \neg exc \Rightarrow \text{new Employee(\emph{employee})})$
	\item exception: $exc := getUserByEmail(employee.email) \neq \text{None}$ \\ $\Rightarrow
		      \text{EmployeeAlreadyExistsException}$
\end{itemize}

\noindent createShopOwner($shopOwner$):
\begin{itemize}
	\item transition: new Employee(shopOwner), save shop owner to employee database
	\item output: $out := (exc = \text{ShopOwnerAlreadyExistsException}$ \\ $\Rightarrow ``\text{User with
			      email address already exists.}" |\ \neg exc \Rightarrow \text{new Employee(\emph{shopOwner})})$
	\item exception: $exc := getUserByEmail(shopOwner.email) \neq \text{None}$ \\ $\Rightarrow
		      \text{ShopOwnerAlreadyExistsException}$
\end{itemize}

\noindent getUserByEmail($e$):
\begin{itemize}
	\item output: $out :=$ A User such that is contains the email, $e$, from the customers or employees, else
	      None.
\end{itemize}

\noindent authorize($email, password$):
\begin{itemize}
	\item output: $out := getUserByEmail(email) = \text{User} \Rightarrow $ \\ $(\text{User} = \text{None}
		      \Rightarrow \text{``User not found."}$ \\ $|\ \text{User.password} = password \Rightarrow
		      \text{AuthorizeReturnType}$ \\ $|\ \text{User.password} \neq password \Rightarrow
		      \text{``Unauthorized"})$
\end{itemize}

\subsubsection{Local Functions}

\noindent hash($s$):
\begin{itemize}
	\item output: $out := \text{hashed value of } s$
\end{itemize}

% \section{MIS of \wss{Module Name}} \label{Module} \wss{Use labels for
% 	cross-referencing}

% \wss{You can reference SRS labels, such as R\ref{R_Inputs}.}

% \wss{It is also possible to use \LaTeX for hypperlinks to external documents.}

% \subsection{Module}

% \wss{Short name for the module}

% \subsection{Uses}

% \subsection{Syntax}

% \subsubsection{Exported Constants}

% \subsubsection{Exported Access Programs}

% \begin{center}
% 	\begin{tabular}{p{2cm} p{4cm} p{4cm} p{2cm}}
% 		\hline
% 		\textbf{Name}    & \textbf{In} & \textbf{Out} & \textbf{Exceptions} \\
% 		\hline
% 		\wss{accessProg} & -           & -            & -                   \\
% 		\hline
% 	\end{tabular}
% \end{center}

% \subsection{Semantics}

% \subsubsection{State Variables}

% \wss{Not all modules will have state variables.  State variables give the module
% 	a memory.}

% \subsubsection{Environment Variables}

% \wss{This section is not necessary for all modules.  Its purpose is to capture
% 	when the module has external interaction with the environment, such as for a
% 	device driver, screen interface, keyboard, file, etc.}

% \subsubsection{Assumptions}

% \wss{Try to minimize assumptions and anticipate programmer errors via
% 	exceptions, but for practical purposes assumptions are sometimes appropriate.}

% \subsubsection{Access Routine Semantics}

% \noindent \wss{accessProg}():
% \begin{itemize}
% 	\item transition: \wss{if appropriate}
% 	\item output: \wss{if appropriate}
% 	\item exception: \wss{if appropriate}
% \end{itemize}

% \wss{A module without environment variables or state variables is unlikely to
% 	have a state transition.  In this case a state transition can only occur if
% 	the module is changing the state of another module.}

% \wss{Modules rarely have both a transition and an output.  In most cases you
% 	will have one or the other.}

% \subsubsection{Local Functions}

% \wss{As appropriate} \wss{These functions are for the purpose of specification.
% 	They are not necessarily something that is going to be implemented
% 	explicitly.  Even if they are implemented, they are not exported; they only
% 	have local scope.}

\newpage

\bibliographystyle{plainnat}
\bibliography{../../../refs/References}

\newpage

\section{Appendix}

\wss{Extra information if required}

\end{document}
