\documentclass[12pt, titlepage]{article}

\usepackage{fullpage}
\usepackage[round]{natbib}
\usepackage{multirow}
\usepackage{booktabs}
\usepackage{tabularx}
\usepackage{graphicx}
\usepackage{float}
\usepackage{hyperref}

\hypersetup{
	colorlinks,
	citecolor=blue,
	filecolor=black,
	linkcolor=red,
	urlcolor=blue
}

%% Comments

\usepackage{color}

\newif\ifcomments\commentstrue %displays comments
%\newif\ifcomments\commentsfalse %so that comments do not display

\ifcomments
\newcommand{\authornote}[3]{\textcolor{#1}{[#3 ---#2]}}
\newcommand{\todo}[1]{\textcolor{red}{[TODO: #1]}}
\else
\newcommand{\authornote}[3]{}
\newcommand{\todo}[1]{}
\fi

\newcommand{\wss}[1]{\authornote{blue}{SS}{#1}} 
\newcommand{\plt}[1]{\authornote{magenta}{TPLT}{#1}} %For explanation of the template
\newcommand{\an}[1]{\authornote{cyan}{Author}{#1}}

%% Common Parts

\newcommand{\progname}{Sayyara}
\newcommand{\authname}{Team 3, Tiny Coders
	\\ Arkin Modi
	\\ Joy Xiao
	\\ Leon So
	\\ Timothy Choy} % AUTHOR NAMES

\usepackage{hyperref}
\hypersetup{colorlinks=true, linkcolor=blue, citecolor=blue, filecolor=blue,
	urlcolor=blue, unicode=false}
\urlstyle{same}

\usepackage{parskip}
\usepackage{geometry}
\geometry{a4paper, portrait, margin=1in}


\newcounter{acnum}
\newcommand{\actheacnum}{AC\theacnum}
\newcommand{\acref}[1]{AC\ref{#1}}

\newcounter{ucnum}
\newcommand{\uctheucnum}{UC\theucnum}
\newcommand{\uref}[1]{UC\ref{#1}}

\newcounter{mnum}
\newcommand{\mthemnum}{M\themnum}
\newcommand{\mref}[1]{M\ref{#1}}

\begin{document}

\title{System Design for \progname{}}
\author{\authname}
\date{\today}

\maketitle

\pagenumbering{roman}

\section{Revision History}

\begin{table}[hp]
	\caption{Revision History} \label{TblRevisionHistory}
	\begin{tabularx}{\textwidth}{llX}
		\toprule
		\textbf{Date}     & \textbf{Developer(s)} & \textbf{Change}                                  \\
		\midrule
		December 28, 2022 & Arkin Modi            & Revision History \& Mark Not Applicable Sections \\
		January 7, 2023   & Joy Xiao              & Introduction \& Purpose                          \\
		January 11, 2023  & Leon So               & Undesired Event Handling                         \\
		\bottomrule
	\end{tabularx}
\end{table}

\newpage

\section{Reference Material}

This section records information for easy reference.

\subsection{Abbreviations and Acronyms}

\begin{tabular}{l l}
	\toprule
	\textbf{symbol} & \textbf{description}            \\
	\midrule
	\progname       & Explanation of program name     \\
	MIS             & Module Interface Specifications \\
	MG              & Module Guide                    \\
	\bottomrule
\end{tabular}

\newpage

\tableofcontents

\newpage

\listoftables

\listoffigures

\newpage

\pagenumbering{arabic}

\section{Introduction}

The following document details the System Design for project Sayyara. Sayyara is a progressive web
application (PWA) which will act as a single platform for independent auto repair shops and vehicle
owners. This platform will allow independent auto repair shops and vehicle owners to interact in a
more efficient and effective manner. Vehicle owners can search for auto repair shops and services;
request quotes for service; book, view, and manage service appointments. On the application, auto
repair shop owners will be able to manage a list of employees; manage a list of service types and
corresponding service appointment availabilities; manage store information such as location, hours
of operation, and contact information. Auto repair shop owners and employees will be able to manage
quotes, service appointments, and work orders from a single application.

Complementary documents include the Module Interface Specifications and Module Guide. The full
documentation and implementation can be found at
\url{https://github.com/arkinmodi/project-sayyara/}.

\section{Purpose}

The purpose of this document is to display the component decomposition of the system and provide
the user interface designs of the software being built. The implementation of the software will be
based off of the designs within this document. The MIS
\url{https://github.com/arkinmodi/project-sayyara/blob/main/docs/Design/SoftDetailedDes/MIS.pdf}
and MG
\url{https://github.com/arkinmodi/project-sayyara/blob/main/docs/Design/SoftArchitecture/MG.pdf}
are also created to give details to the software architecture and detailed component breakdowns for
the project.

\section{Scope}

\wss{Include a figure that show the System Context (showing the boundary between
	your system and the environment around it.)}

\section{Project Overview}

\subsection{Normal Behaviour}

\subsection{Undesired Event Handling}

Undesired events will be handled both client-side and server-side.

On the client-side, if an unexpected event arises or the application enters a bad state, the
application will reset to a safe state. For example, if a user attempts to access a route that they
are not authorized to access, they will be either be redirected to an appropriate route, prompted
to login, or an error page will be displayed with instructions to return to the home page. Input
forms will also include input validation to ensure only properly formed data is handled. If the
user attempts to input invalid data, the form field will reset and form submission will be blocked.
The user will be prompted to enter a valid input value in the form field. Similarly, various user
actions and inputs that may pose cause that the application to enter an undesirable state will be
validated before updating the application state.

On the server-side, each API will return a response with the appropriate error status code and
message. Subsequently, the client will have logic to gracefully handle unsuccessful responses and
status codes, preventing the system from entering an undesirable state. Inputs will also be
validated on the server-side by parsing the input data using defined schemas. This will ensure data
integrity and prevents the undesirable data from entering the workflows or database.

\subsection{Component Diagram}

\subsection{Connection Between Requirements and Design} \label{SecConnection}

\wss{The intention of this section is to document decisions that are made
	``between'' the requirements and the design.  To satisfy some requirements,
	design decisions need to be made.  Rather than make these decisions implicit,
	they are explicitly recorded here.  For instance, if a program has security
	requirements, a specific design decision may be made to satisfy those
	requirements with a password.}

\section{System Variables}
\subsection{Monitored Variables}
N/A

\subsection{Controlled Variables}
N/A

\subsection{Constants Variables}
N/A

\section{User Interfaces}

\wss{Design of user interface for software and hardware.  Attach an appendix if
	needed. Drawings, Sketches, Figma}

\section{Design of Hardware}
N/A

\section{Design of Electrical Components}
N/A

\section{Design of Communication Protocols}
N/A

\section{Timeline}

\wss{Schedule of tasks and who is responsible}

% \bibliographystyle {plainnat}
% \bibliography{../../../refs/References}

\newpage

\section{Appendix}

\subsection{Interface}

\wss{Include additional information related to the appearance of, and
	interaction with, the user interface}

\subsection{Reflection}

The information in this section will be used to evaluate the team members on the graduate attribute
of Problem Analysis and Design. Please answer the following questions:

\begin{enumerate}
	\item What are the limitations of your solution? Put another way, given unlimited resources, what could
	      you do to make the project better? (LO\_ProbSolutions)
	\item Give a brief overview of other design solutions you considered. What are the benefits and tradeoffs
	      of those other designs compared with the chosen design? From all the potential options, why did you
	      select documented design? (LO\_Explores)
\end{enumerate}

\end{document}
