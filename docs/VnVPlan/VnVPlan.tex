\documentclass[12pt, titlepage]{article}

\usepackage{booktabs}
\usepackage{tabularx}
\usepackage{hyperref}
\hypersetup{
    colorlinks,
    citecolor=blue,
    filecolor=black,
    linkcolor=red,
    urlcolor=blue
}
\usepackage[round]{natbib}

%% Comments

\usepackage{color}

\newif\ifcomments\commentstrue %displays comments
%\newif\ifcomments\commentsfalse %so that comments do not display

\ifcomments
\newcommand{\authornote}[3]{\textcolor{#1}{[#3 ---#2]}}
\newcommand{\todo}[1]{\textcolor{red}{[TODO: #1]}}
\else
\newcommand{\authornote}[3]{}
\newcommand{\todo}[1]{}
\fi

\newcommand{\wss}[1]{\authornote{blue}{SS}{#1}} 
\newcommand{\plt}[1]{\authornote{magenta}{TPLT}{#1}} %For explanation of the template
\newcommand{\an}[1]{\authornote{cyan}{Author}{#1}}

%% Common Parts

\newcommand{\progname}{Sayyara}
\newcommand{\authname}{Team 3, Tiny Coders
	\\ Arkin Modi
	\\ Joy Xiao
	\\ Leon So
	\\ Timothy Choy} % AUTHOR NAMES

\usepackage{hyperref}
\hypersetup{colorlinks=true, linkcolor=blue, citecolor=blue, filecolor=blue,
	urlcolor=blue, unicode=false}
\urlstyle{same}

\usepackage{parskip}
\usepackage{geometry}
\geometry{a4paper, portrait, margin=1in}


\begin{document}

\title{Project Title: System Verification and Validation Plan for \progname{}}
\author{\authname}
\date{\today}

\maketitle

\pagenumbering{roman}

\section{Revision History}

\begin{table}[hp]
	\caption{Revision History} \label{TblRevisionHistory}
	\begin{tabularx}{\textwidth}{llX}
		\toprule
		\textbf{Date}    & \textbf{Developer(s)} & \textbf{Change}                          \\
		\midrule
		October 26, 2022 & Arkin Modi            & Added Software Validation Plan           \\
		October 27, 2022 & Joy Xiao              & Add Testing Team and Design Verification \\
		October 28, 2022 & Timothy Choy          & Add Relevant Documentation               \\
		\bottomrule
	\end{tabularx}
\end{table}

\newpage

\tableofcontents

\listoftables
\wss{Remove this section if it isn't needed}

\listoffigures
\wss{Remove this section if it isn't needed}

\newpage

\section{Symbols, Abbreviations and Acronyms}

\renewcommand{\arraystretch}{1.2}
\begin{tabular}{l l}
	\toprule
	\textbf{symbol} & \textbf{description} \\
	\midrule
	T               & Test                 \\
	\bottomrule
\end{tabular}\\

\wss{symbols, abbreviations or acronyms --- you can simply reference the SRS
	\citep{SRS} tables, if appropriate}

\wss{Remove this section if it isn't needed}

\newpage

\pagenumbering{arabic}

This document ... \wss{provide an introductory blurb and roadmap of the Verification and Validation
	plan}

\section{General Information}

\subsection{Summary}

\wss{Say what software is being tested.  Give its name and a brief overview of
	its general functions.}

\subsection{Objectives}

\wss{State what is intended to be accomplished.  The objective will be around
	the qualities that are most important for your project.  You might have
	something like: ``build confidence in the software correctness,''
	``demonstrate adequate usability.'' etc.  You won't list all of the qualities,
	just those that are most important.}

\subsection{Relevant Documentation}

Below are the documents that will be referred to in the V\&V Plan.

\begin{itemize}
	\item \href{https://github.com/arkinmodi/project-sayyara/blob/main/docs/SRS/SRS.pdf}{Software
		      Requirements Specification (SRS)}
	\item \href{https://github.com/arkinmodi/project-sayyara/blob/main/docs/Design/MG/MG.pdf}{Module
		      Guide (MG)}
	\item \href{https://github.com/arkinmodi/project-sayyara/blob/main/docs/Design/MIS/MIS.pdf}{Module
		      Interface Specification (MIS)}
\end{itemize}

\wss{Reference relevant documentation.  This will definitely include your SRS
	and your other project documents (design documents, like MG, MIS, etc).  You
	can include these even before they are written, since by the time the project
	is done, they will be written.}

\citet{SRS}

\section{Plan}

\wss{Introduce this section.   You can provide a roadmap of the sections to
	come.}

\subsection{Verification and Validation Team}

The verification and validation team will consist of the core developers (Joy Xiao, Tim Choy, Leon
So, Arkin Modi), as well as the course instructor and TAs.

The developers are responsible for coming up with tests with suitable edge cases to evaluate the
correctness of Sayyara. The developers will all be responsible for writing and executing all test
cases listed in the document and taking note of the results. The developers will ensure that
Sayyara passes all tests after performing the tests and making any necessary updates.

\subsection{SRS Verification Plan}

\wss{List any approaches you intend to use for SRS verification.  This may include
	ad hoc feedback from reviewers, like your classmates, or you may plan for
	something more rigorous/systematic.}

\wss{Maybe create an SRS checklist?}

\subsection{Design Verification Plan}

Design verification will be done by the core developers of the project. The design will also be
reviewed by the TAs of the course. The design of the system will be verified by going through the
requirements from the
\href{https://github.com/arkinmodi/project-sayyara/blob/main/docs/SRS/SRS.pdf}{Software
	Requirements Specification (SRS)} and determining whether the outputs correspond with the expected
inputs. The verification will also be done by going through the
\href{https://github.com/arkinmodi/project-sayyara/blob/main/docs/Design/MG/MG.pdf}{Module Guide
	(MG)} and
\href{https://github.com/arkinmodi/project-sayyara/blob/main/docs/Design/MIS/MIS.pdf}{Module
	Interface Specification (MIS)} checklist and ensure that all the modules are completed and fulfill
the corresponding requirements.

\subsection{Verification and Validation Plan Verification Plan}

\wss{The verification and validation plan is an artifact that should also be verified.}

\wss{The review will include reviews by your classmates}

\wss{Create a checklists?}

\subsection{Implementation Verification Plan}

\wss{You should at least point to the tests listed in this document and the unit
	testing plan.}

\wss{In this section you would also give any details of any plans for static verification of
	the implementation.  Potential techniques include code walkthroughs, code
	inspection, static analyzers, etc.}

\subsection{Automated Testing and Verification Tools}

\wss{What tools are you using for automated testing.  Likely a unit testing
	framework and maybe a profiling tool, like ValGrind.  Other possible tools
	include a static analyzer, make, continuous integration tools, test coverage
	tools, etc.  Explain your plans for summarizing code coverage metrics.
	Linters are another important class of tools.  For the programming language
	you select, you should look at the available linters.  There may also be tools
	that verify that coding standards have been respected, like flake9 for
	Python.}

\wss{If you have already done this in the development plan, you can point to
	that document.}

\wss{The details of this section will likely evolve as you get closer to the
	implementation.}

\subsection{Software Validation Plan}

The plan for validating the software and the requirements shall be to conduct review session with
the stakeholders. These review sessions shall focus on the business events and user flows as
defined in the
\href{https://github.com/arkinmodi/project-sayyara/blob/main/docs/SRS/SRS.pdf}{Software
	Requirements Specification (SRS)}.

\section{System Test Description}

\subsection{Tests for Functional Requirements}

\wss{Subsets of the tests may be in related, so this section is divided into
	different areas.  If there are no identifiable subsets for the tests, this
	level of document structure can be removed.}

\wss{Include a blurb here to explain why the subsections below
	cover the requirements.  References to the SRS would be good here.}

\subsubsection{Area of Testing1}

\wss{It would be nice to have a blurb here to explain why the subsections below
	cover the requirements.  References to the SRS would be good here.  If a section
	covers tests for input constraints, you should reference the data constraints
	table in the SRS.}

\paragraph{Title for Test}

\begin{enumerate}

	\item{test-id1\\}

	Control: Manual versus Automatic

	Initial State:

	Input:

	Output: \wss{The expected result for the given inputs}

	Test Case Derivation: \wss{Justify the expected value given in the Output field}

	How test will be performed:

	\item{test-id2\\}

	Control: Manual versus Automatic

	Initial State:

	Input:

	Output: \wss{The expected result for the given inputs}

	Test Case Derivation: \wss{Justify the expected value given in the Output field}

	How test will be performed:

\end{enumerate}

\subsubsection{Area of Testing2}

...

\subsection{Tests for Nonfunctional Requirements}

\wss{The nonfunctional requirements for accuracy will likely just reference the
	appropriate functional tests from above.  The test cases should mention
	reporting the relative error for these tests.  Not all projects will
	necessarily have nonfunctional requirements related to accuracy}

\wss{Tests related to usability could include conducting a usability test and
	survey.  The survey will be in the Appendix.}

\wss{Static tests, review, inspections, and walkthroughs, will not follow the
	format for the tests given below.}

\subsubsection{Area of Testing1}

\paragraph{Title for Test}

\begin{enumerate}

	\item{test-id1\\}

	Type: Functional, Dynamic, Manual, Static etc.

	Initial State:

	Input/Condition:

	Output/Result:

	How test will be performed:

	\item{test-id2\\}

	Type: Functional, Dynamic, Manual, Static etc.

	Initial State:

	Input:

	Output:

	How test will be performed:

\end{enumerate}

\subsubsection{Area of Testing2}

...

\subsection{Traceability Between Test Cases and Requirements}

\wss{Provide a table that shows which test cases are supporting which
	requirements.}

\section{Unit Test Description}

\wss{Reference your MIS (detailed design document) and explain your overall
	philosophy for test case selection.}
\wss{This section should not be filled in until after the MIS (detailed design
	document) has been completed.}

\subsection{Unit Testing Scope}

\wss{What modules are outside of the scope.  If there are modules that are
	developed by someone else, then you would say here if you aren't planning on
	verifying them.  There may also be modules that are part of your software, but
	have a lower priority for verification than others.  If this is the case,
	explain your rationale for the ranking of module importance.}

\subsection{Tests for Functional Requirements}

\wss{Most of the verification will be through automated unit testing.  If
	appropriate specific modules can be verified by a non-testing based
	technique.  That can also be documented in this section.}

\subsubsection{Module 1}

\wss{Include a blurb here to explain why the subsections below cover the module.
	References to the MIS would be good.  You will want tests from a black box
	perspective and from a white box perspective.  Explain to the reader how the
	tests were selected.}

\begin{enumerate}

	\item{test-id1\\}

	Type: \wss{Functional, Dynamic, Manual, Automatic, Static etc. Most will be automatic}

	Initial State:

	Input:

	Output: \wss{The expected result for the given inputs}

	Test Case Derivation: \wss{Justify the expected value given in the Output field}

	How test will be performed:

	\item{test-id2\\}

	Type: \wss{Functional, Dynamic, Manual, Automatic, Static etc. Most will be automatic}

	Initial State:

	Input:

	Output: \wss{The expected result for the given inputs}

	Test Case Derivation: \wss{Justify the expected value given in the Output field}

	How test will be performed:

	\item{...\\}

\end{enumerate}

\subsubsection{Module 2}

...

\subsection{Tests for Nonfunctional Requirements}

\wss{If there is a module that needs to be independently assessed for
	performance, those test cases can go here.  In some projects, planning for
	nonfunctional tests of units will not be that relevant.}

\wss{These tests may involve collecting performance data from previously
	mentioned functional tests.}

\subsubsection{Module ?}

\begin{enumerate}

	\item{test-id1\\}

	Type: \wss{Functional, Dynamic, Manual, Automatic, Static etc. Most will be automatic}

	Initial State:

	Input/Condition:

	Output/Result:

	How test will be performed:

	\item{test-id2\\}

	Type: Functional, Dynamic, Manual, Static etc.

	Initial State:

	Input:

	Output:

	How test will be performed:

\end{enumerate}

\subsubsection{Module ?}

...

\subsection{Traceability Between Test Cases and Modules}

\wss{Provide evidence that all of the modules have been considered.}

\bibliographystyle{plainnat}

\bibliography{../../refs/References}

\newpage

\section{Appendix}

This is where you can place additional information.

\subsection{Symbolic Parameters}

The definition of the test cases will call for SYMBOLIC\_CONSTANTS. Their values are defined in
this section for easy maintenance.

\subsection{Usability Survey Questions?}

\wss{This is a section that would be appropriate for some projects.}

\newpage{}
\section*{Appendix --- Reflection}

The information in this section will be used to evaluate the team members on the graduate attribute
of Lifelong Learning. Please answer the following questions:

\newpage{}
\section*{Appendix --- Reflection}

The information in this section will be used to evaluate the team members on the graduate attribute
of Lifelong Learning. Please answer the following questions:

\begin{enumerate}
	\item What knowledge and skills will the team collectively need to acquire to successfully complete the
	      verification and validation of your project? Examples of possible knowledge and skills include
	      dynamic testing knowledge, static testing knowledge, specific tool usage etc. You should look to
	      identify at least one item for each team member.
	\item For each of the knowledge areas and skills identified in the previous question, what are at least
	      two approaches to acquiring the knowledge or mastering the skill? Of the identified approaches,
	      which will each team member pursue, and why did they make this choice?
\end{enumerate}

\end{document}
