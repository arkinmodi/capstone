\documentclass{article}

\usepackage{booktabs}
\usepackage{tabularx}
\usepackage{float}

\title{Development Plan\\\progname}

\author{\authname}

\date{}

%% Comments

\usepackage{color}

\newif\ifcomments\commentstrue %displays comments
%\newif\ifcomments\commentsfalse %so that comments do not display

\ifcomments
\newcommand{\authornote}[3]{\textcolor{#1}{[#3 ---#2]}}
\newcommand{\todo}[1]{\textcolor{red}{[TODO: #1]}}
\else
\newcommand{\authornote}[3]{}
\newcommand{\todo}[1]{}
\fi

\newcommand{\wss}[1]{\authornote{blue}{SS}{#1}} 
\newcommand{\plt}[1]{\authornote{magenta}{TPLT}{#1}} %For explanation of the template
\newcommand{\an}[1]{\authornote{cyan}{Author}{#1}}

%% Common Parts

\newcommand{\progname}{Sayyara}
\newcommand{\authname}{Team 3, Tiny Coders
	\\ Arkin Modi
	\\ Joy Xiao
	\\ Leon So
	\\ Timothy Choy} % AUTHOR NAMES

\usepackage{hyperref}
\hypersetup{colorlinks=true, linkcolor=blue, citecolor=blue, filecolor=blue,
	urlcolor=blue, unicode=false}
\urlstyle{same}

\usepackage{parskip}
\usepackage{geometry}
\geometry{a4paper, portrait, margin=1in}


\begin{document}

\begin{table}[hp]
\caption{Revision History} \label{TblRevisionHistory}
\begin{tabularx}{\textwidth}{llX}
\toprule
\textbf{Date} & \textbf{Developer(s)} & \textbf{Change}\\
\midrule
September 23, 2022 & Arkin Modi & Create project scheduling section\\
September 23, 2022 & Arkin Modi & Create workflow plan\\
\bottomrule
\end{tabularx}
\end{table}

\newpage

\maketitle

\wss{Put your introductory blurb here.}

\section{Team Meeting Plan}

\section{Team Communication Plan}

\section{Team Member Roles}

\section{Workflow Plan}

% \begin{itemize}
%   \item How will you be using git, including branches, pull request, etc.?
%   \item How will you be managing issues, including template issues, issue
%   classification, etc.?
% \end{itemize}

\subsection{Development Workflow}

The team will be using Git as their version control system and the repository will be hosted
publicly on GitHub. The most stable up-to-date version of our application will be on the ``main''
branch. The team will update the main branch using a feature branch workflow. The feature branch
workflow entails the developer to:

\begin{enumerate}
  \item Create a new feature branch
  \item Push changes to the new branch
  \item Open a pull request against the main branch
  \item Link the pull request to an issue (if appropriate)
  \item Add ``needs review'' label (if ready for review)
  \item Assign assignees and reviewers
  \item Pass any pull request pipelines in our CI/CD setup
  \item Receive reviews from other team members
  \item Address any comments or change requests
  \item Merge or close the pull request
\end{enumerate}

To prevent naming conflicts, the team will follow a branch naming convention of
``$<$GitHub username$>$/$<$branch name$>$''. The merge strategy used for pull requests will be
``squash and merge''. Using this merge strategy increases traceability on the main branch's commit
history by having every commit represented by a working/passing pull request and prevents adding
broken commits to the main branch's commit history.

All team members are listed as code owners of everything in the repository (defined in
.github/CODEOWNERS). GitHub will read this and automatically assigned all team members to every pull
request opened in the repository. Every pull request should receive, at minimum, one review before
merging.

\subsection{Project Management Workflow}

The team will be using GitHub's issue tracker, GitHub Projects, and GitHub Milestones to manage the
project. The issue tracker will track work that needs to be done. For each task that isn't trivial,
an issue will be open to track the contribution and progress. Examples of trivial tasks include
spelling corrections, dependency version updates, base repository template updates. Each issue will
also be assigned to team member who will then assume responsibility for said issue. Issues may also
be assigned to a GitHub Milestone. A milestone will represent a deliverable's deadline and will be
an additional way to track progress towards a deadline.

Labels, as described in Table \ref{issueLabels}, will used to organize issues. The GitHub Project
board will be used to visualize the progress of each issue and track overall progress.

\begin{table}[H]
  \centering
  \caption{Issue Labels}
  \vspace{5pt}
  \begin{tabular}{|p{0.2\textwidth}|p{0.7\textwidth}|}
      \hline
      \textbf{Name} & \textbf{Description}\\
      \hline
      documentation & These issues are related to documentation or reports for course deliverables.\\
      \hline
      Epic & These issues are tracking a collection of other issues. For example, an issues a report
      has sub-issues for each section.\\
      \hline
      meeting needed & A team meeting is needed to discuss details related to the issue.\\
      \hline
      meeting notes & This issue contains notes from a completed meeting.\\
      \hline
      needs breakdown & This issue needs further breakdown into smaller issues/steps.\\
      \hline
      tech foundation & This issue is not contributing towards any course deliverables.\\
      \hline
  \end{tabular}

  \label{issueLabels}
\end{table}


\section{Proof of Concept Demonstration Plan}

What is the main risk, or risks, for the success of your project?  What will you
demonstrate during your proof of concept demonstration to convince yourself that
you will be able to overcome this risk?

\section{Technology}

\begin{itemize}
\item Specific programming language
\item Specific linter tool (if appropriate)
\item Specific unit testing framework
\item Investigation of code coverage measuring tools
\item Specific plans for Continuous Integration (CI), or an explanation that CI
  is not being done
\item Specific performance measuring tools (like Valgrind), if
  appropriate
\item Libraries you will likely be using?
\item Tools you will likely be using?
\end{itemize}

\section{Coding Standard}

\section{Project Scheduling}

% \wss{How will the project be scheduled?}

The project schedule will primarily follow the course calendar with development of the application
starting in October 2022. Progress of the project will be tracked using a GitHub Project board. The
deliverables due date will act as our milestones and will be tracked with GitHub Milestones.

Decomposition of large tasks into smaller tasks for documentation deliverables will done by breaking
down the reports by section. Sections will be determined by consulting provided templates, rubrics,
and examples. Breaking large development tasks into smaller tasks will be done by first researching
and scoping out how the task should be architected. Following this initial architecture, smaller
tasks will be created.

Responsibility/ownership of tasks will be decided as a team during weekly meetings or offline
communications. It is upon each member to communicate their availability and capable workload.
Ideally, the work shall be equally distributed over the course of the project, but a shift in
workloads on an individual deliverable basis is expected. If a member is unable to complete a task
they are assigned, it is their responsibility to communicate this to the team and have the task
reassigned. The expectation is each member shall contribute an average of 10 hours per week.

\end{document}
