\documentclass{article}

\usepackage{booktabs}
\usepackage{tabularx}
\usepackage{float}

\title{Development Plan\\\progname}

\author{\authname}

\date{}

%% Comments

\usepackage{color}

\newif\ifcomments\commentstrue %displays comments
%\newif\ifcomments\commentsfalse %so that comments do not display

\ifcomments
\newcommand{\authornote}[3]{\textcolor{#1}{[#3 ---#2]}}
\newcommand{\todo}[1]{\textcolor{red}{[TODO: #1]}}
\else
\newcommand{\authornote}[3]{}
\newcommand{\todo}[1]{}
\fi

\newcommand{\wss}[1]{\authornote{blue}{SS}{#1}} 
\newcommand{\plt}[1]{\authornote{magenta}{TPLT}{#1}} %For explanation of the template
\newcommand{\an}[1]{\authornote{cyan}{Author}{#1}}

%% Common Parts

\newcommand{\progname}{Sayyara}
\newcommand{\authname}{Team 3, Tiny Coders
	\\ Arkin Modi
	\\ Joy Xiao
	\\ Leon So
	\\ Timothy Choy} % AUTHOR NAMES

\usepackage{hyperref}
\hypersetup{colorlinks=true, linkcolor=blue, citecolor=blue, filecolor=blue,
	urlcolor=blue, unicode=false}
\urlstyle{same}

\usepackage{parskip}
\usepackage{geometry}
\geometry{a4paper, portrait, margin=1in}


\begin{document}

\maketitle

\begin{table}[hp]
	\caption{Revision History} \label{TblRevisionHistory}
	\begin{tabularx}{\textwidth}{llX}
		\toprule
		\textbf{Date}      & \textbf{Developer(s)} & \textbf{Change}                                  \\
		\midrule
		September 23, 2022 & Timothy Choy          & Add roles and responsibilities                   \\
		September 23, 2022 & Arkin Modi            & Create project scheduling section                \\
		September 23, 2022 & Arkin Modi            & Create workflow plan                             \\
		September 23, 2022 & Arkin Modi            & Create technology and code standard sections     \\
		September 25, 2022 & Timothy Choy          & Create team meeting plan                         \\
		September 25, 2022 & Timothy Choy          & Create team communication plan                   \\
		September 25, 2022 & Timothy Choy          & Create proof of concept demo plan                \\
		September 26, 2022 & Timothy Choy          & Add introductory blurb                           \\
		March 8, 2023      & Joy Xiao              & Change component library                         \\
		March 18, 2023     & Arkin Modi            & Update weekly meeting times                      \\
		March 26, 2023     & Arkin Modi            & Add static analysis and code documentation tools \\
		\bottomrule
	\end{tabularx}
\end{table}

This document outlines the team's development plan moving forward. This document will include the
plans for meeting, and communications between team members and the capstone project supervisor.
Furthermore, this document also includes each members' roles and responsibilities. On the technical
side, this document will state the technologies used, workflow plan, coding standards and what Tiny
Coders will be demonstrating in the proof of concept demo.

\section{Team Meeting Plan}

The team will have weekly team meetings every Friday at 10:00am for the fall semester, and every
Wednesday at 2:30pm for the winter semester. These meetings will be held on Discord. The purpose of
these meetings are to realign the team to the current deliverables and to assign tasks to each team
member. An agenda of every team meeting will be written as a GitHub issue, and tagged with the
meeting number and meeting date. Each meeting will be archived and can be easily accessed by
searching for issues with the ``meeting notes'' tag. Additional meetings with the supervisor will
be held on an as-needed basis.

\section{Team Communication Plan}

The team will communicate primarily through Discord, Messenger, and GitHub Issues. Discord and
Messenger will be used for quick communication between the team members, with the added benefit
that the supervisor will also use Discord. Team meetings will also be hosted on Discord. For
general communication, organization, and job delegation, GitHub Issues will be used. The team has
set up GitHub Projects to further organize issues into categories such as ``Todo'', ``In
Progress'', ``Review/QA'', and ``Done''.

\section{Team Member Roles and Responsibilities}

Every member will have the following responsibilities:

\begin{itemize}
	\item Contribute to full stack development, not just components the member is leading
	\item Write tests for their code
	\item Contribute to relevant documentation
	\item Code reviews through approvals in pull requests
	\item Follow guidelines set for updating the main branch, stated in CONTRIBUTING.md
	\item Follow the development workflow
	\item Attend meetings, or giving adequate time to let the team know if they cannot attend
	\item Communicate with the supervisor of the project
\end{itemize}

Furthermore, each member has specific roles as shown in the table below. Though the team has
assigned specific people to each role, they are subject to change and can be moved around from
member to member.

\begin{table}[H]
	\centering
	\caption{Specific Member Roles}
	\vspace{5pt}
	\begin{tabular}{|p{0.2\textwidth}|p{0.3\textwidth}|p{0.4\textwidth}|}
		\hline
		\textbf{Team Member} & \textbf{Role(s)}                   & \textbf{Responsibilities}                             \\
		\hline
		Arkin Modi           & Point of Contact                   & The liaison between the professor/TAs
		and the team                                                                                                      \\
		\cline{2-3}
		                     & Issue Tracker Manager              & Manages issue tracker, keeps the issues
		organized                                                                                                         \\
		\cline{2-3}
		                     & Database, Deployment \& CI/CD Lead & Manages the database component of the application, as
		well as the deployment and CI/CD process                                                                          \\
		\hline
		Joy Xiao             & Front End Lead                     & Manages the front end components of the application   \\
		\cline{2-3}
		                     & Note Taker                         & Takes meeting minutes for meetings, if necessary      \\
		\hline
		Leon So              & Server Lead                        & Manages the server components of the application      \\
		\cline{2-3}
		                     & Note Taker                         & Takes meeting minutes for meetings, if necessary      \\
		\hline
		Timothy Choy         & Meeting Chair                      & Organizes team meetings and creates meeting agendas   \\
		\cline{2-3}
		                     & Front End Lead                     & Manages the front end components of the application   \\
		\hline
	\end{tabular}
\end{table}

\section{Workflow Plan}

\subsection{Development Workflow}

The team will be using Git as their version control system and the repository will be hosted
publicly on GitHub. The most stable up-to-date version of the application will be on the ``main''
branch. The team will update the main branch using a feature branch workflow. The feature branch
workflow entails the developer to:

\begin{enumerate}
	\item Create a new feature branch
	\item Push changes to the new branch
	\item Open a pull request against the main branch
	\item Link the pull request to an issue (if appropriate)
	\item Add ``needs review'' label (if ready for review)
	\item Assign assignees and reviewers
	\item Pass any pull request pipelines in the CI/CD setup
	\item Receive reviews from other team members
	\item Address any comments or change requests
	\item Merge or close the pull request
\end{enumerate}

To prevent naming conflicts, the team will follow a branch naming convention of ``$<$GitHub
username$>$/$<$branch name$>$''. The merge strategy used for pull requests will be ``squash and
merge''. Using this merge strategy increases traceability on the main branch's commit history by
having every commit represented by a working/passing pull request and prevents adding broken
commits to the main branch's commit history.

All team members are listed as code owners of everything in the repository (defined in
.github/CODEOWNERS). GitHub will read this and automatically assigned all team members to every
pull request opened in the repository. Every pull request should receive, at minimum, one review
before merging.

\subsection{Project Management Workflow}

The team will be using GitHub's issue tracker, GitHub Projects, and GitHub Milestones to manage the
project. The issue tracker will track work that needs to be done. For each task that isn't trivial,
an issue will be open to track the contribution and progress. Examples of trivial tasks include
spelling corrections, dependency version updates, base repository template updates. Each issue will
also be assigned to team member who will then assume responsibility for said issue. Issues may also
be assigned to a GitHub Milestone. A milestone will represent a deliverable's deadline and will be
an additional way to track progress towards a deadline.

Labels, as described in Table \ref{issueLabels}, will be used to organize issues. The GitHub
Project board will be used to visualize the progress of each issue and track overall progress.

\begin{table}[H]
	\centering
	\caption{Issue Labels}
	\vspace{5pt}
	\begin{tabular}{|p{0.2\textwidth}|p{0.7\textwidth}|}
		\hline
		\textbf{Name}   & \textbf{Description}                                                                       \\
		\hline
		documentation   & These issues are related to documentation or reports for course deliverables.              \\
		\hline
		Epic            & These issues are tracking a collection of other issues. For example, an issue for a report
		has sub-issues for each section.                                                                             \\
		\hline
		meeting needed  & A team meeting is needed to discuss details related to the issue.                          \\
		\hline
		meeting notes   & This issue contains notes from a completed meeting.                                        \\
		\hline
		needs breakdown & This issue needs further breakdown into smaller issues/steps.                              \\
		\hline
		tech foundation & This issue is not directly contributing towards any course deliverables.                   \\
		\hline
	\end{tabular}

	\label{issueLabels}
\end{table}

\section{Proof of Concept Demonstration Plan}

There are several risks that will determine the success or failure of the project. The main risks
are:

\begin{itemize}
	\item Communication between the modules of the PWA
	\item Responsiveness of the PWA to different devices and screen resolutions
	\item Peering problems between multiple editors (e.g., race conditions)
	\item Integration testing
	\item The team is unfamiliar with PWAs, as this will be the team's first time working on one
\end{itemize}

To prove that the team will be able to overcome these risks and have a successful project, the team
will be showing several aspects in the proof of concept demo:

\begin{itemize}
	\item Develop two modules of the PWA to show communication between the modules in the application works
	\item Demo the application on several devices (e.g., an iPhone, an Android, and on a computer) to show
	      that the application is responsive
	\item Demo an example of two devices attempting to schedule an appointment at the same time, and see the
	      result
\end{itemize}

\section{Technology}

The project will be written in TypeScript using the Next.js framework. Next.js is a React framework
that provides both a frontend and a backend interface. The team is comfortable with React, and with
both the core Next.js team and the core React team recommending Next.js for ``production-ready
projects'', the team sees this as a good choice. Tiny Coders will be using NPM as the package
manager and the packages next-auth and next-pwa. NextAuth is a package built for Next.js to quickly
enable authentication with many out of the box adaptors for providers and databases. Next-PWA is a
package to quickly enable the creating of Progressive Web App (PWA)'s in Next.js. The team has
decided to create a PWA since this was one of the key application attributes that the supervisor
required.

For the user interface, the team has chosen to use PrimeReact as the component library alongside
CSS. A team member has previous experience with this library and recommended using it.

The team will be using Prisma for ORM (Object-relational mapping) and MySQL as the database
management system. The team has decided to use a SQL database due to the data having relations.
MySQL was specifically chosen due to past experience within the team. In a NoSQL database,
relations are harder to maintain and/or possibly inefficient to query. SQL databases also provide
the assurances of ACID compliance, fewer consistency issues, and enforced field constraints. The
use of Prisma is due to past experience within the team and a strong reputation within the web
development world.

The Jest Testing Framework and React Testing Library will be used for unit tests, integration
tests, and code coverage. Most of the team does not have much experience with frontend testing,
therefore the team has decided to go with the libraries recommended by Next.js. The team has
decided against using any end-to-end testing frameworks (e.g., Cypress, Playwright) due to lack of
domain knowledge within the team and time constraints.

Git and GitHub will be used for version control and hosting a centralized online repository. This
was chosen due to course requirements and past experience.

Documentation will be written in a mix of \LaTeX{} and Markdown. Code documentation will be written
inline with the source code as docstrings and will follow JSDoc annotations. The docstrings will be
rendered into HTML documentation using TypeDoc.

With high performance not being a primary goal for this project, the team will not use be using any
performance benchmarking tools. The team will be maintaining a Google Lighthouse score of greater
than or equal to 90\%.

All CI/CD pipelines will be created using GitHub Actions. This was chosen due to GitHub Actions
having great integration with GitHub, a generous free and pro tier, past experience within the
team, and ease of use.

The development environment for any developer is a highly personal decision and while the team will
not be enforcing the use any specific local dev tooling, the team will be recommending to all
developers to use Visual Studio Code for easy interoperability between developers. The team will
also have a list of recommended extensions and setting for Visual Studio Code available in the
``.vscode'' folder. The team will also be using Docker for easy setup of external dependencies
(e.g., TeX Live, MySQL).

\section{Coding Standard}

To ensure consistent code styling and code quality, the application code will be formatted using
Prettier and ESLint. Prettier is an industry standard code formatter and widely agreed upon default
configuration. ESLint is an industry standard static analysis tool for TypeScript.

For \LaTeX{} docs, the team will be using latexindent.pl. latexindent.pl is a popular \LaTeX{} code
formatter. The primary use will be for managing whitespace to ensure consistent readability in
source files. The primary reason behind this choice was latexindent.pl's code formatter
capabilities to automatically apply fixes.

\section{Project Scheduling}

The project schedule will primarily follow the course calendar with development of the application
starting in October 2022. Progress of the project will be tracked using a GitHub Project board. The
deliverables due date will act as milestones and will be tracked with GitHub Milestones.

Decomposition of large tasks into smaller tasks for documentation deliverables will be done by
breaking down the reports by section. Sections will be determined by consulting provided templates,
rubrics, and examples. Breaking large development tasks into smaller tasks will be done by first
researching and scoping out how the task should be architected. Following this initial
architecture, smaller tasks will be created.

Responsibility/ownership of tasks will be decided as a team during weekly meetings or offline
communications. It is upon each member to communicate their availability and capable workload.
Ideally, the work shall be equally distributed over the course of the project, but a shift in
workloads on an individual deliverable basis is expected. If a member is unable to complete a task
they are assigned, it is their responsibility to communicate this to the team and have the task
reassigned. The expectation is each member shall contribute an average of 10 hours per week.

\end{document}
