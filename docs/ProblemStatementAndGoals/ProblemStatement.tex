\documentclass{article}

\usepackage{tabularx}
\usepackage{booktabs}

\title{Problem Statement and Goals\\\progname}

\author{\authname}

\date{}

%% Comments

\usepackage{color}

\newif\ifcomments\commentstrue %displays comments
%\newif\ifcomments\commentsfalse %so that comments do not display

\ifcomments
\newcommand{\authornote}[3]{\textcolor{#1}{[#3 ---#2]}}
\newcommand{\todo}[1]{\textcolor{red}{[TODO: #1]}}
\else
\newcommand{\authornote}[3]{}
\newcommand{\todo}[1]{}
\fi

\newcommand{\wss}[1]{\authornote{blue}{SS}{#1}} 
\newcommand{\plt}[1]{\authornote{magenta}{TPLT}{#1}} %For explanation of the template
\newcommand{\an}[1]{\authornote{cyan}{Author}{#1}}

%% Common Parts

\newcommand{\progname}{Sayyara}
\newcommand{\authname}{Team 3, Tiny Coders
	\\ Arkin Modi
	\\ Joy Xiao
	\\ Leon So
	\\ Timothy Choy} % AUTHOR NAMES

\usepackage{hyperref}
\hypersetup{colorlinks=true, linkcolor=blue, citecolor=blue, filecolor=blue,
	urlcolor=blue, unicode=false}
\urlstyle{same}

\usepackage{parskip}
\usepackage{geometry}
\geometry{a4paper, portrait, margin=1in}


\begin{document}

\maketitle

\begin{table}[hp]
\caption{Revision History} \label{TblRevisionHistory}
\begin{tabularx}{\textwidth}{llX}
\toprule
\textbf{Date} & \textbf{Developer(s)} & \textbf{Change}\\
\midrule
September 20, 2022 & Joy Xiao & Problem statement and stakeholders\\
September 23, 2022 & Leon So & Update problem\\
September 23, 2022 & Joy Xiao & Add project goals\\
... & ... & ...\\
\bottomrule
\end{tabularx}
\end{table}

\section{Problem Statement}

\wss{You should check your problem statement with the
\href{https://github.com/smiths/capTemplate/blob/main/docs/Checklists/ProbState-Checklist.pdf}
{problem statement checklist}.}
\wss{You can change the section headings, as long as you include the required information.}

\subsection{Problem}
Independent auto repair shops do not have an efficient way of reaching and interacting with new customers.
Currently, many independent shop owners rely on word-of-mouth referrals as a main channel to acquiring new customers. 
Independent auto repair shops are also spending a significant amount of their time on administrative work such as 
managing appointments and providing quotes. As a result, independent auto repair shops have a difficult time 
competing with larger repair shops which have dedicated systems and services in place.\\

On the other hand, customers do not have an effective way to find and compare auto repair shops. 
Currently, one of the only ways to compare repair shops is by manually searching or reaching out to repair shops one-by-one. 
This process can often be repetitive and time-consuming.

\subsection{Inputs and Outputs}

\wss{Characterize the problem in terms of ``high level'' inputs and outputs.  
Use abstraction so that you can avoid details.}

\subsection{Stakeholders}
- Automobile owners
- Independent shop mechanics

\subsection{Environment}

\wss{Hardware and software}

\section{Goals}
\begin{itemize}
\subsection{Work Orders}
\item Shop owners can create, update, view, filter, search work orders. Any details such as estimated time, parts used, and costs related to the work order and showing the technican assigned to each order. Allow the shop owner to send work order receipts to the customer and request payment.
\subsection{Quotes}
\item Shop owners can give quotes to customer requests. The shop owners will be able to chat with the vehicle owners and respond to requests easily through the application.
\item Vechicle owners can request a quote on their job. This will allow vehicle owners to easily get quotes from multiple mechanics and compare prices.
\subsection{Appointments}
\item Shop owners can create, view, edit appointments. Shop owners can set their availability on their calendar and view appointments scheduled.
\item Vehicle owners can search for auto shops and book an appointment at available auto shops. They can select the time and date when they are available and schedule a time with an auto shop that is available during that time.
\end{itemize}

\section{Stretch Goals}

\end{document}
