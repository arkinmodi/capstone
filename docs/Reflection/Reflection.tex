\documentclass{article}

\usepackage{tabularx}
\usepackage{booktabs}

\title{Reflection Report on \progname}

\author{\authname}

\date{}

%% Comments

\usepackage{color}

\newif\ifcomments\commentstrue %displays comments
%\newif\ifcomments\commentsfalse %so that comments do not display

\ifcomments
\newcommand{\authornote}[3]{\textcolor{#1}{[#3 ---#2]}}
\newcommand{\todo}[1]{\textcolor{red}{[TODO: #1]}}
\else
\newcommand{\authornote}[3]{}
\newcommand{\todo}[1]{}
\fi

\newcommand{\wss}[1]{\authornote{blue}{SS}{#1}} 
\newcommand{\plt}[1]{\authornote{magenta}{TPLT}{#1}} %For explanation of the template
\newcommand{\an}[1]{\authornote{cyan}{Author}{#1}}

%% Common Parts

\newcommand{\progname}{Sayyara}
\newcommand{\authname}{Team 3, Tiny Coders
	\\ Arkin Modi
	\\ Joy Xiao
	\\ Leon So
	\\ Timothy Choy} % AUTHOR NAMES

\usepackage{hyperref}
\hypersetup{colorlinks=true, linkcolor=blue, citecolor=blue, filecolor=blue,
	urlcolor=blue, unicode=false}
\urlstyle{same}

\usepackage{parskip}
\usepackage{geometry}
\geometry{a4paper, portrait, margin=1in}


\begin{document}

\maketitle

\section{Changes in Response to Feedback}

\subsection{SRS and Hazard Analysis}
We received TA feedback to add more details to our requirements. This was updated for our
functional and nonfunctional requirements to give more details, which makes our requirements more
focused. We also had feedback to make our requirements more specific. For example, specifying
constraints for inputs and adding a list/table of figures. As a result, we updated the SRS to
address this feedback. After the Rev 0 Demo, we got TA feedback to add a cancellation reason when a
customer or a shop cancels an appointment. This was added for our Rev 1 Demo and added as a
requirement in the SRS. Another change we made was to reduce the scope of our project, which
resulted in changing the SRS. The scope change was made in agreement with the project supervisor,
and only features that were not necessarily essential for the application were taken out. We also
improved the introduction for the Hazard Analysis in response to TA feedback.

\subsection{Design and Design Documentation}
Based on TA feedback, we have revised our System Design documentation to include descriptions for
mock-ups and clarification on the system variables. Additionally, the Module Interface
Specification document was updated to reflect changes during development and changes from feedback
for the Verification and Validation Plan and Report.

\subsection{VnV Plan and Report}
The TA gave us feedback to add more details to our tests to make the tests more specific. We also
conducted a usability survey with customers and auto shop owners. From doing the usability survey,
there were a few feature requests from the test users which were some bug fixes, adding clickable
links on the shop profile, password requirements, and filter buttons for the search page. We also
demoed the project to our project supervisor, which was very impressed with our progress. Our
project supervisor was also one of our test users for the project, and he thought the project was
very well done. These upgrades were added for the Rev 1 Demo and updated in our VnV plan and
report. We also updated the VnV report to include more specific changes due to testing. For
example, we included the reason why the test failed and what changes were made to address the
failing test case.

\section{Design Iteration (LO11)}
Initially, our team took the requirements provided by our project supervisor, Nabeel, and created
formal specifications which were later translated to mock-ups. Our team took an incremental and
iterative approach, where we would continuously collect feedback from our stakeholders, including
the project supervisor, the course TA and instructor, auto repair shop mechanics, and other
potential users. During implementation and prototyping (i.e., mockups), we realized critical flaws
in the workflows and user experience envisioned by the project supervisor, and we made design
changes to address these flaws. Examples include implementing a live quote system, adding a common
landing page for the web application, and organizing features into dashboards. We also incorporated
feedback from the Rev 0 demo. For example, we incorporated the ability to cancel appointments, the
ability to edit appointments, and added notifications (in the form of toasts) to confirm various
user actions have triggered an event.

\section{Design Decisions (LO12)}
Due to time limitations and resource limitations (i.e., our team only had 4 team members), we had
to work with our project supervisor to determine an appropriate scope for the project. This led to
the necessary sacrifice of some features we would have loved to design and implement. We also had
to make critical assumptions due to time constraints, which would allow us to limit the scope. For
example, we made the assumption that each vehicle owner would only have one vehicle. For the scope
of this project, we simplified the scheduling process to only support one appointment at a time.
However, in reality, a shop would likely have multiple service bays and employees, allowing for
concurrent and parallel appointments. As a result, the product is missing some features necessary
for commercialization of the product, such as an employee invitation process using email services,
the ability to edit customer, vehicle, and shop metadata.

\section{Economic Considerations (LO23)}
Yes, there is certainly a market for our product. The market is segmented into two segments: (1)
auto repair shop owners, and (2) anyone who owns a vehicle in Canada. This product provides many
features and functionality which allow both vehicle owners and auto repair shops to more easily
interact. Furthermore, this allows auto repair shops to streamline their administrative tasks and
operations, allowing them to spend more time repairing cars to drive and generate revenue.

In terms of marketing the product, various marketing campaigns and materials would be required. The
content of these marketing materials should be tailored towards two main audiences: auto repair
shop owners and vehicle owners. These are typically the decision makers who would decide whether to
onboard to the application.

The product would be free to use for vehicle owners. In the future, the application would process
payments, which would allow for a revenue stream through transaction fees. For shop owners, there
will be a subscription model with three tiers, each unlocking a separate set of features. The free
tier will include a set of basic functionality, which includes the current set of features. The
second tier would include additional features such as analytic dashboards to monitor work progress,
appointments, quotes, etc. The second tier would cost \$5 per month. The third tier would include
additional features which includes a set of advertising and marketing features within the platform,
such as featured spotlight (to display the spot at the top of results for a specified number of
days per month), in-app advertisements, email lists and campaigns features, and marketing analytics
(e.g., number of clicks). The third tier would cost \$10 per month.

There are currently no active users, as this product is scoped as an initial prototype for the
project supervisor. It was built with the intention for the product to be extended further to
support commercialization after the completion of this capstone project and course.

\section{Reflection on Project Management (LO24)}

\subsection{How Does Your Project Management Compare to Your Development Plan}
Yes, we followed the development plan at met during the weekly meeting times. We followed the team
communication plan as well. We used most of the technologies listed in the development plan. There
was a component library mentioned in the development plan, which we had to switch to a different
one after trying it for the proof of concept demo. The team chose a new component library and
updated the development plan.

\subsection{What Went Well?}
Having weekly meetings as stated on the development plan was very helpful for our project
management. This made us set aside a dedicated time to figure out the next steps and work together
as a group. Some technologies that worked well were our CI/CD pipeline for building documents, unit
& integration tests, and for performing code styling checks. This made processes more efficient
when writing documents, ensured that there were no major issues with each change by running a
testing check, and enforced specific code styling for consistency.

\subsection{What Went Wrong?}
Nothing major went wrong as all team members committed and followed through with the plan. All team
members were also communicative when any issues had arisen. However, our team could have improved
on the time management, specifically in the delivery and implementation of features for each
milestone. This would have allowed for more time to polish the application, and iterate based on
testing and feedback. In terms of technology, our team found the set of technologies used supported
the project management processes well, and we did not experience any difficulty or challenges
relating to technology.

\subsection{What Would you Do Differently Next Time?}
If given the opportunity, we would have loved to collect more user feedback to further improve our
design and user experience. Next time, we would plan for more time in order to get iterative
feedback from test users. We would also revise the scope by planning to include front-end
integration tests in the scope, as well as refine the set of features for the initial project. This
would entail reducing the set of features within the scope and punting some features as stretch
goals. This would allow us to polish and refine the most critical features. To improve on time
management, our team would set soft deadlines for every feature. This would allow our team to set
expectations, leave a time buffer for polish, and allow the team to better coordinate tasks. One
additional project management change we would make is to incorporate a more defined and strict
style guide. Given that there are many ways to do things, for example, defining a function in
TypeScript, structuring classes and components in React, etc., it would be beneficial to have an
agreed upon style to enforce uniformity and consistency among code.

\end{document}
