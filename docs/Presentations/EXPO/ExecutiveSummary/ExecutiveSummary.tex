\documentclass{article}

\usepackage{booktabs}
\usepackage{tabularx}

\title{
    McMaster Engineering Capstone Expo\\
    Department of Computing and Software\\
    Executive Summary\\
    \progname
}

\author{\authname}

\date{}

%% Comments

\usepackage{color}

\newif\ifcomments\commentstrue %displays comments
%\newif\ifcomments\commentsfalse %so that comments do not display

\ifcomments
\newcommand{\authornote}[3]{\textcolor{#1}{[#3 ---#2]}}
\newcommand{\todo}[1]{\textcolor{red}{[TODO: #1]}}
\else
\newcommand{\authornote}[3]{}
\newcommand{\todo}[1]{}
\fi

\newcommand{\wss}[1]{\authornote{blue}{SS}{#1}} 
\newcommand{\plt}[1]{\authornote{magenta}{TPLT}{#1}} %For explanation of the template
\newcommand{\an}[1]{\authornote{cyan}{Author}{#1}}

%% Common Parts

\newcommand{\progname}{Sayyara}
\newcommand{\authname}{Team 3, Tiny Coders
	\\ Arkin Modi
	\\ Joy Xiao
	\\ Leon So
	\\ Timothy Choy} % AUTHOR NAMES

\usepackage{hyperref}
\hypersetup{colorlinks=true, linkcolor=blue, citecolor=blue, filecolor=blue,
	urlcolor=blue, unicode=false}
\urlstyle{same}

\usepackage{parskip}
\usepackage{geometry}
\geometry{a4paper, portrait, margin=1in}


\begin{document}

\maketitle

Sayyara is a progressive web application (PWA) which will act as a single platform for independent
automotive repair shops and vehicle owners. The application can be accessed by users through a web
browser on both desktop and mobile devices. This platform will allow repair shops and vehicle
owners to interact in a more efficient and effective manner. Vehicle owners can search for
automotive repair shops and services based on a variety of search filters; request quotes for
service; book, view, and manage service appointments. On the application, shop owners will have
full shop management capabilities such as: adding and managing a list of employees; managing a list
of service types and corresponding service appointment availabilities; managing store information
such as location, hours of operation, and contact information. Automotive repair shop owners and
employees will be able to manage quotes, service appointments, and work orders from a single
application. Ultimately, Sayyara will significantly improve the auto repair experience for both
independent automotive repair shops and vehicle owners.

Sayyara is built using a variety of development tools and infrastructure. The website is a React
application built using the Next.js framework for the front-end and back-end, Prisma for the object
relational mapping (ORM) (i.e., database communications), and the Bing Maps API for powering the
geolocation services. The website is hosted on Vercel using a serverless architecture and the Node
16 runtime. The database chosen to hold all the data is MySQL, a relational database, and is hosted
on PlanetScale. The code base is version controlled using Git and is hosted on GitHub. The
continuous integration and continuous delivery (CI/CD) is run on the GitHub Actions platform, where
all tests and code quality checks are run on all additions to the code base. All tests were written
using the Jest testing framework.

Sayyara is built using the culmination of all the software engineering design processes taught
throughout the Software Engineering program. The process began with initial problem identification
in the ``Problem Statement and Goals'' report. This outlined the problem proposed by the
supervisor, Nabeel Ibrahim, at a high level. Details on how the team plans to effective work
together were outlined in the ``Development Plan''. The problem was then broken down into a set of
clearly defined requirements in the ``Software Requirements Specification''. Next, any potential
hazards and risks were reviewed in the ``Hazard Analysis''. Following this, a plan was created for
how the team will verify that the application meets the requirements in the ``Verification and
Validation Plan''. Design plans on the concrete implementation of the solution were planned out in
a set of three documents: ``Module Guide'', ``Module Interface Specification'', and ``System
Design''. The solution was then built, and the verification plan was executed, with the results
documents in the ``Verification and Validation Report''. While the engineering process outlined
here is linear, in reality, moving to the next did not imply that the previous step was set in
stone. As new information and ideas were discovered throughout the process, past documents would be
revised and updated.

\end{document}
