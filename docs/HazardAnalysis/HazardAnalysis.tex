\documentclass{article}

\usepackage{booktabs}
\usepackage{tabularx}
\usepackage{hyperref}
\usepackage{pdflscape}
\usepackage{float}
\usepackage{adjustbox}

\hypersetup{
    colorlinks=true,       % false: boxed links; true: colored links
    linkcolor=red,          % color of internal links (change box color with linkbordercolor)
    citecolor=green,        % color of links to bibliography
    filecolor=magenta,      % color of file links
    urlcolor=cyan           % color of external links
}

\title{Hazard Analysis\\\progname}

\author{\authname}

\date{\today}

%% Comments

\usepackage{color}

\newif\ifcomments\commentstrue %displays comments
%\newif\ifcomments\commentsfalse %so that comments do not display

\ifcomments
\newcommand{\authornote}[3]{\textcolor{#1}{[#3 ---#2]}}
\newcommand{\todo}[1]{\textcolor{red}{[TODO: #1]}}
\else
\newcommand{\authornote}[3]{}
\newcommand{\todo}[1]{}
\fi

\newcommand{\wss}[1]{\authornote{blue}{SS}{#1}} 
\newcommand{\plt}[1]{\authornote{magenta}{TPLT}{#1}} %For explanation of the template
\newcommand{\an}[1]{\authornote{cyan}{Author}{#1}}

%% Common Parts

\newcommand{\progname}{Sayyara}
\newcommand{\authname}{Team 3, Tiny Coders
	\\ Arkin Modi
	\\ Joy Xiao
	\\ Leon So
	\\ Timothy Choy} % AUTHOR NAMES

\usepackage{hyperref}
\hypersetup{colorlinks=true, linkcolor=blue, citecolor=blue, filecolor=blue,
	urlcolor=blue, unicode=false}
\urlstyle{same}

\usepackage{parskip}
\usepackage{geometry}
\geometry{a4paper, portrait, margin=1in}


\begin{document}

\maketitle
\thispagestyle{empty}

~\newpage

\pagenumbering{roman}

\begin{table}[hp]
	\caption{Revision History} \label{TblRevisionHistory}
	\begin{tabularx}{\textwidth}{llX}
		\toprule
		\textbf{Date}    & \textbf{Developer(s)} & \textbf{Change}                               \\
		\midrule
		October 13, 2022 & Arkin Modi            & Create Failure Mode and Effect Analysis table \\
		October 14, 2022 & Joy Xiao              & Introduction                                  \\
		\bottomrule
	\end{tabularx}
\end{table}

~\newpage

\tableofcontents

~\newpage

\pagenumbering{arabic}

\wss{You are free to modify this template.}

\section{Introduction}
This document outlines the hazard analysis of Sayyara. The definition of hazard is any property or
condition in the system along with conditions in the environment that may cause harm or damage.
This definition is from Nancy Leveson's work. The hazards for Sayyara include security and usage
hazards such as protecting personal information, database failures, and having no internet
connection.

\section{Scope and Purpose of Hazard Analysis}
The scope of the hazard analysis is to identify any hazards that may arise when using the
application, their causes, coming up with steps to eliminate or mitigate the effect of the hazard.
The purpose of the hazard analysis is to pinpoint areas where hazards may arise and their effects
and come up with mitigation steps. Through completing the hazard analysis, safety and security
requirements will be developed early in the design process to minimize the risk of having hazards
occur without plans in place to reduce or mitigate the effects.

\section{System Boundaries and Components}
The system consists of:
\begin{enumerate}
	\item The application's front-end and back-end components in the major categories:
	      \begin{itemize}
		      \item Authentication
		      \item Appointments
		      \item Quotes
		      \item Work Orders
		      \item Shop Profile
		      \item Services
		      \item Shop Employees
	      \end{itemize}
	\item The database being used which will store all of application's data
\end{enumerate}

\section{Critical Assumptions}

\wss{These assumptions that are made about the software or system.  You should
	minimize the number of assumptions that remove potential hazards.  For instance,
	you could assume a part will never fail, but it is generally better to include
	this potential failure mode.}

There are no critical assumptions made.

\newpage
\begin{landscape}
	\section{Failure Mode and Effect Analysis}
	\begin{table}[H]
		\centering
		\caption{Failure Mode and Effect Analysis Table}
		\begin{adjustbox}{width=\paperwidth}
			\begin{tabular}{c|c|c|c|c|c|c}
				\hline
				\textbf{Component} & \textbf{Failure Modes} & \textbf{Effects of Failure} & \textbf{Causes of Failure} & \textbf{Recommended Action} & \textbf{SR} & \textbf{Ref.} \\
				\hline
			\end{tabular}
		\end{adjustbox}
	\end{table}
\end{landscape}

\section{Safety and Security Requirements}

\wss{Newly discovered requirements.  These should also be added to the SRS.  (A
	rationale design process how and why to fake it.)}

\section{Roadmap}

\wss{Which safety requirements will be implemented as part of the capstone timeline?
	Which requirements will be implemented in the future?}

\end{document}
