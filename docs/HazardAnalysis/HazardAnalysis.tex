\documentclass{article}

\usepackage{booktabs}
\usepackage{tabularx}
\usepackage{hyperref}
\usepackage{pdflscape}
\usepackage{float}
\usepackage{adjustbox}
\usepackage{multirow}
\usepackage{enumitem}
\usepackage{longtable}

\hypersetup{
	colorlinks=true,		% false: boxed links; true: colored links
	linkcolor=red,			% color of internal links (change box color with linkbordercolor)
	citecolor=green,		% color of links to bibliography
	filecolor=magenta,	% color of file links
	urlcolor=cyan				% color of external links
}

\title{Hazard Analysis\\\progname}

\author{\authname}

\date{\today}

%% Comments

\usepackage{color}

\newif\ifcomments\commentstrue %displays comments
%\newif\ifcomments\commentsfalse %so that comments do not display

\ifcomments
\newcommand{\authornote}[3]{\textcolor{#1}{[#3 ---#2]}}
\newcommand{\todo}[1]{\textcolor{red}{[TODO: #1]}}
\else
\newcommand{\authornote}[3]{}
\newcommand{\todo}[1]{}
\fi

\newcommand{\wss}[1]{\authornote{blue}{SS}{#1}} 
\newcommand{\plt}[1]{\authornote{magenta}{TPLT}{#1}} %For explanation of the template
\newcommand{\an}[1]{\authornote{cyan}{Author}{#1}}

%% Common Parts

\newcommand{\progname}{Sayyara}
\newcommand{\authname}{Team 3, Tiny Coders
	\\ Arkin Modi
	\\ Joy Xiao
	\\ Leon So
	\\ Timothy Choy} % AUTHOR NAMES

\usepackage{hyperref}
\hypersetup{colorlinks=true, linkcolor=blue, citecolor=blue, filecolor=blue,
	urlcolor=blue, unicode=false}
\urlstyle{same}

\usepackage{parskip}
\usepackage{geometry}
\geometry{a4paper, portrait, margin=1in}


\begin{document}

\maketitle
\thispagestyle{empty}

~\newpage

\pagenumbering{roman}

\begin{table}[hp]
	\caption{Revision History} \label{TblRevisionHistory}
	\begin{tabularx}{\textwidth}{llX}
		\toprule
		\textbf{Date}    & \textbf{Developer(s)} & \textbf{Change}                                                                \\
		\midrule
		October 13, 2022 & Arkin Modi            & Create Failure Mode and Effect Analysis table                                  \\
		October 16, 2022 & Arkin Modi            & Fill in FMEA table for Work Orders, Shop Profile, Services, and Shop Employees \\
		\bottomrule
	\end{tabularx}
\end{table}

~\newpage

\tableofcontents

~\newpage

\pagenumbering{arabic}

\wss{You are free to modify this template.}

\section{Introduction}

\wss{You can include your definition of what a hazard is here.}

\section{Scope and Purpose of Hazard Analysis}

\section{System Boundaries and Components}

\section{Critical Assumptions}

\wss{These assumptions that are made about the software or system.  You should
	minimize the number of assumptions that remove potential hazards.  For instance,
	you could assume a part will never fail, but it is generally better to include
	this potential failure mode.}

\newpage
\begin{landscape}
	\section{Failure Mode and Effect Analysis}
	\begin{longtable}{|p{0.15\textwidth}|p{0.15\textwidth}|p{0.25\textwidth}|p{0.25\textwidth}|p{0.35\textwidth}|p{0.1\textwidth}|p{0.05\textwidth}|}
		\caption{Failure Mode and Effect Analysis Table}                                                               \\
		\hline
		\multicolumn{1}{|c|}{\textbf{Component}}
		 & \multicolumn{1}{|c|}{\textbf{Failure Modes}}
		 & \multicolumn{1}{|c|}{\textbf{Effects of Failure}}
		 & \multicolumn{1}{|c|}{\textbf{Causes of Failure}}
		 & \multicolumn{1}{|c|}{\textbf{Recommended Action}}
		 & \multicolumn{1}{|c|}{\textbf{SR}}
		 & \multicolumn{1}{|c|}{\textbf{Ref.}}                                                                         \\
		\hline
		\multirow{2}{*}{Work Orders}
		 & Work Order is missing
		 & Customer and Employees will not know any of the work that has been done for a specific job
		 & \begin{enumerate}[label=\alph*., leftmargin=*]
			   \item Database failure
		   \end{enumerate}
		 & \begin{enumerate}[label=\alph*., leftmargin=*]
			   \item Regular and automatic database backups/snapshots and allow shop owners to request rollbacks
		   \end{enumerate}
		 & \begin{enumerate}[label=\alph*., leftmargin=*]
			   \item
		   \end{enumerate}
		 & H5-1                                                                                                        \\
		\cline{2-7}
		~
		 & Work Order is missing detailed information
		 & Customer and Employees will not know all of the work that has been done for a specific job
		 & \begin{enumerate}[label=\alph*., leftmargin=*]
			   \item Database failure
		   \end{enumerate}
		 & \begin{enumerate}[label=\alph*., leftmargin=*]
			   \item Refer to H5-1a
		   \end{enumerate}
		 & \begin{enumerate}[label=\alph*., leftmargin=*]
			   \item
		   \end{enumerate}
		 & H5-2                                                                                                        \\
		\hline
		Shop Profile
		 & Unable to find shop details
		 & Customers will not be able to see the information about a specific shop (e.g., address, phone number, etc.)
		 & \begin{enumerate}[label=\alph*., leftmargin=*]
			   \item Database failure
		   \end{enumerate}
		 & \begin{enumerate}[label=\alph*., leftmargin=*]
			   \item Refer to H5-1a
		   \end{enumerate}
		 & \begin{enumerate}[label=\alph*., leftmargin=*]
			   \item
		   \end{enumerate}
		 & H6-1                                                                                                        \\
		\hline
		\multirow{2}{*}{Services}
		 & Unable to find a service
		 & Customers and Employees will not be able to see what services are offered by the shop
		 & \begin{enumerate}[label=\alph*., leftmargin=*]
			   \item Database failure
		   \end{enumerate}
		 & \begin{enumerate}[label=\alph*., leftmargin=*]
			   \item Refer to H5-1a
		   \end{enumerate}
		 & \begin{enumerate}[label=\alph*., leftmargin=*]
			   \item
		   \end{enumerate}
		 & H7-1                                                                                                        \\
		\cline{2-7}
		~
		 & Unable to find a service's details
		 & Customers and Employees will not be able to see the service's details (e.g., price, estimated time, etc.)
		 & \begin{enumerate}[label=\alph*., leftmargin=*]
			   \item Database failure
		   \end{enumerate}
		 & \begin{enumerate}[label=\alph*., leftmargin=*]
			   \item Refer to H5-1a
		   \end{enumerate}
		 & \begin{enumerate}[label=\alph*., leftmargin=*]
			   \item
		   \end{enumerate}
		 & H7-1                                                                                                        \\
		\hline
		Shop Employees
		 & A former employee joins the shop account
		 & The former employee can view sensitive information and perform unauthorized actions
		 & \begin{enumerate}[label=\alph*., leftmargin=*]
			   \item A former employee accepts their invite link to join the shop as an employee after their employment
			         has been terminated
		   \end{enumerate}
		 & \begin{enumerate}[label=\alph*., leftmargin=*]
			   \item Invite links should expire after a set period of time
			   \item Invite links should only be able to be accepts once
			   \item Shop owners should be able to revoke access to any employee
		   \end{enumerate}
		 & \begin{enumerate}[label=\alph*., leftmargin=*]
			   \item
		   \end{enumerate}
		 & H8-1                                                                                                        \\
		\hline
	\end{longtable}
\end{landscape}

\section{Safety and Security Requirements}

\wss{Newly discovered requirements.  These should also be added to the SRS.  (A
	rationale design process how and why to fake it.)}

\section{Roadmap}

\wss{Which safety requirements will be implemented as part of the capstone timeline?
	Which requirements will be implemented in the future?}

\wss{The Roadmap is where you can explain which safety requirements will be implemented during the
	course, and which will be ``postponed'' until after the course.}

\end{document}
