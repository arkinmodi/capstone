\documentclass[12pt, titlepage]{article}

\usepackage{booktabs}
\usepackage{tabularx}
\usepackage{hyperref}
\hypersetup{
    colorlinks,
    citecolor=black,
    filecolor=black,
    linkcolor=red,
    urlcolor=blue
}
\usepackage[round]{natbib}

%% Comments

\usepackage{color}

\newif\ifcomments\commentstrue %displays comments
%\newif\ifcomments\commentsfalse %so that comments do not display

\ifcomments
\newcommand{\authornote}[3]{\textcolor{#1}{[#3 ---#2]}}
\newcommand{\todo}[1]{\textcolor{red}{[TODO: #1]}}
\else
\newcommand{\authornote}[3]{}
\newcommand{\todo}[1]{}
\fi

\newcommand{\wss}[1]{\authornote{blue}{SS}{#1}} 
\newcommand{\plt}[1]{\authornote{magenta}{TPLT}{#1}} %For explanation of the template
\newcommand{\an}[1]{\authornote{cyan}{Author}{#1}}

%% Common Parts

\newcommand{\progname}{Sayyara}
\newcommand{\authname}{Team 3, Tiny Coders
	\\ Arkin Modi
	\\ Joy Xiao
	\\ Leon So
	\\ Timothy Choy} % AUTHOR NAMES

\usepackage{hyperref}
\hypersetup{colorlinks=true, linkcolor=blue, citecolor=blue, filecolor=blue,
	urlcolor=blue, unicode=false}
\urlstyle{same}

\usepackage{parskip}
\usepackage{geometry}
\geometry{a4paper, portrait, margin=1in}


\begin{document}

\title{Verification and Validation Report: \progname}
\author{\authname}
\date{\today}

\maketitle

\pagenumbering{roman}

\section{Revision History}

\begin{table}[hp]
	\caption{Revision History} \label{TblRevisionHistory}
	\begin{tabularx}{\textwidth}{llX}
		\toprule
		\textbf{Date}     & \textbf{Developer(s)} & \textbf{Change}               \\
		\midrule
		February 22, 2023 & Arkin Modi            & Add Automated Testing Section \\
		\bottomrule
	\end{tabularx}
\end{table}
\newpage

\section{Symbols, Abbreviations and Acronyms}

\renewcommand{\arraystretch}{1.2}
\begin{tabular}{l l}
	\toprule
	\textbf{symbol} & \textbf{description}   \\
	\midrule
	T               & Test                   \\
	CI              & Continuous Integration \\
	CD              & Continuous Deployment  \\
	\bottomrule
\end{tabular}\\

\newpage

\tableofcontents

\listoftables %if appropriate

\listoffigures %if appropriate

\newpage

\pagenumbering{arabic}

This document ...

\section{Functional Requirements Evaluation}

\section{Nonfunctional Requirements Evaluation}

\subsection{Usability}

\subsection{Performance}

\subsection{etc.}

\section{Unit Testing}

\section{Changes Due to Testing}

\section{Automated Testing}

% The automated testing section can summarize the experience with the use of automated testing tools.

All automated testing was carried out by Jest, a JavaScript testing framework. Each module has its
own corresponding test file located within the ``test'' folder. The ``test'' folder has the same
structure as the ``src'' folder, creating a one-to-one mapping of a module and its corresponding
test. For example, the module ``src/server/services/userService.ts'' has its tests located in
``test/server/services/userService.ts''. There are two Jest test runs, with the only difference
being whether the testing environment is connected to a real database or not (i.e., whether the
database needs to be seeded or mocked in each test case). As part of the CI/CD pipeline, the full
test suite is run and information regarding pass/fail and code coverage is reported on every pull
request to the main branch and to every push to the main branch.

\section{Trace to Requirements}

\section{Trace to Modules}

\section{Code Coverage Metrics}

\bibliographystyle{plainnat}

\bibliography{../../refs/References}

\newpage{}

\section{Appendix}
\subsection{Reflection}

The information in this section will be used to evaluate the team members on the graduate attribute
of Reflection. Please answer the following question:

\begin{enumerate}
	\item In what ways was the Verification and Validation (VnV) Plan different from the activities that were
	      actually conducted for VnV? If there were differences, what changes required the modification in
	      the plan? Why did these changes occur? Would you be able to anticipate these changes in future
	      projects? If there weren't any differences, how was your team able to clearly predict a feasible
	      amount of effort and the right tasks needed to build the evidence that demonstrates the required
	      quality? (It is expected that most teams will have had to deviate from their original VnV Plan.)
\end{enumerate}

\end{document}
