\documentclass[12pt, titlepage]{article}

\usepackage{booktabs}
\usepackage{tabularx}
\usepackage{hyperref}
\hypersetup{
    colorlinks,
    citecolor=black,
    filecolor=black,
    linkcolor=red,
    urlcolor=blue
}
\usepackage[round]{natbib}

%% Comments

\usepackage{color}

\newif\ifcomments\commentstrue %displays comments
%\newif\ifcomments\commentsfalse %so that comments do not display

\ifcomments
\newcommand{\authornote}[3]{\textcolor{#1}{[#3 ---#2]}}
\newcommand{\todo}[1]{\textcolor{red}{[TODO: #1]}}
\else
\newcommand{\authornote}[3]{}
\newcommand{\todo}[1]{}
\fi

\newcommand{\wss}[1]{\authornote{blue}{SS}{#1}} 
\newcommand{\plt}[1]{\authornote{magenta}{TPLT}{#1}} %For explanation of the template
\newcommand{\an}[1]{\authornote{cyan}{Author}{#1}}

%% Common Parts

\newcommand{\progname}{Sayyara}
\newcommand{\authname}{Team 3, Tiny Coders
	\\ Arkin Modi
	\\ Joy Xiao
	\\ Leon So
	\\ Timothy Choy} % AUTHOR NAMES

\usepackage{hyperref}
\hypersetup{colorlinks=true, linkcolor=blue, citecolor=blue, filecolor=blue,
	urlcolor=blue, unicode=false}
\urlstyle{same}

\usepackage{parskip}
\usepackage{geometry}
\geometry{a4paper, portrait, margin=1in}


\begin{document}

\title{Verification and Validation Report: \progname}
\author{\authname}
\date{\today}

\maketitle

\pagenumbering{roman}

\section{Revision History}

\begin{table}[hp]
	\caption{Revision History} \label{TblRevisionHistory}
	\begin{tabularx}{\textwidth}{llX}
		\toprule
		\textbf{Date}     & \textbf{Developer(s)} & \textbf{Change}               \\
		\midrule
		February 22, 2023 & Arkin Modi            & Add Automated Testing Section \\
		\bottomrule
	\end{tabularx}
\end{table}
\newpage

\section{Symbols, Abbreviations and Acronyms}

\renewcommand{\arraystretch}{1.2}
\begin{tabular}{l l}
	\toprule
	\textbf{symbol} & \textbf{description}   \\
	\midrule
	T               & Test                   \\
	CI              & Continuous Integration \\
	CD              & Continuous Deployment  \\
	\bottomrule
\end{tabular}\\

\newpage

\tableofcontents

\listoftables %if appropriate

\listoffigures %if appropriate

\newpage

\pagenumbering{arabic}

This document ...

\section{Functional Requirements Evaluation}

\subsubsection{Services}

This section will contains tests covering the ``Services'' requirements defined in the
\href{https://github.com/arkinmodi/project-sayyara/blob/main/docs/SRS/SRS.pdf}{SRS}.

\begin{enumerate}

	\item \textbf{FRT-BE28-1}

	      Initial State: Shop Owner has a shop profile

	      Input: Shop Owner adds available services

	      Expected Output: Shop services are sent to the database and displayed on the shop profile

	      Results: Passed

	\item \textbf{FRT-BE29-1}

	      Initial State: Active Customer Account

	      Input: Enters specific auto service in the search bar

	      Expected Output: Auto shops with relevant services displayed with service details

	      Results: Fail (does not include service details)

	\item \textbf{FRT-BE29-2}

	      Initial State: Shop Owner/Shop Employee searches for auto repair or maintenance services

	      Input: Enters specific auto service in the search bar

	      Expected Output: Auto shops with relevant services displayed with service details

	      Results: Fail (does not include service details)

	\item \textbf{FRT-BE30-1}

	      Initial State: Shop Owner with services listed on their shop profile

	      Input: Updates the details of a service listed on their profile

	      Expected Output: The service is updated with the entered details

	      Results: Passed

	\item \textbf{FRT-BE31-1}

	      Initial State: Shop Owner with services listed on their shop profile

	      Input: Deletes a service listed on their shop profile

	      Expected Output: Service is removed from the shop profile. Other services that were not deleted
	      continue to be listed on the shop profile

	      Results: Passed

\end{enumerate}

\subsubsection{Appointments}

This section will contain tests covering the ``Appointments'' requirements defined in the
\href{https://github.com/arkinmodi/project-sayyara/blob/main/docs/SRS/SRS.pdf}{SRS}.

\begin{enumerate}

	\item \textbf{FRT-BE4-1}

	      Initial State: Customer account which received a quote from an auto shop

	      Input: Enters service request information and selects an available appointment time slot

	      Expected Output: A new appointment request is created with service details linked to the quote.
	      Shop Owner/Employee Accounts will receive the appointment request

	      Results: Passed

	\item \textbf{FRT-BE4-2}

	      Initial State: Customer account selects a canned job at an auto shop

	      Input: Enters service request information and selects an available appointment time slot

	      Expected Output: A new appointment request is created with service details. Shop Owner/Employee
	      Accounts will receive the appointment request

	      Results: Passed

	\item \textbf{FRT-BE4-3}

	      Initial State: Shop Owner/Employee Account with appointment to be scheduled

	      Input: Enters customer information, service request information and selects an available
	      appointment time slot

	      Expected Output: A new appointment request is created with service details

	      Results:  Failed (do not have this functionality)

	\item \textbf{FRT-BE4-4}

	      Initial State: Shop Owner/Employee Account with appointment requests

	      Input: Accept an appointment request

	      Expected Output: Appointment booking displayed in calendar for Shop Owner/Employee accounts and
	      Customer account

	      Results: Failed (no calendar but the appointment is displayed)

	\item \textbf{FRT-BE5-1}

	      Initial State: Customer Account has a service appointment scheduled

	      Input: Selects an appointment to edit and selects a new available time slot for the appointment

	      Expected Output: A new appointment request is made with the new time slot. Shop Owner/Employee
	      Accounts will receive the appointment request

	      Results: Failed (cannot edit an appointment but can cancel and reschedule)

	\item \textbf{FRT-BE5-2}

	      Initial State: Shop Owner/Employee Account has a service appointment scheduled

	      Input: Selects an appointment to edit and selects a new available time slot for the appointment

	      Expected Output: The appointment booking is updated to the new time slot

	      Results: Failed (cannot edit an appointment but can cancel and reschedule) 

	\item \textbf{FRT-BE5-3}

	      Initial State: Shop Owner/Employee Account has a service appointment scheduled

	      Input: Selects an appointment to edit service details

	      Expected Output: The appointment booking is updated with the updated service details

	      Results: Passed

	\item \textbf{FRT-BE6-1}

	      Initial State: Customer Account has a service appointment scheduled

	      Input: Selects an appointment to cancel

	      Expected Output: The appointment booking is cancelled and removed from the customer and shop
	      owner/employee schedules

	      Results: Passed

	\item \textbf{FRT-BE6-2}

	      Initial State: Shop Owner/Shop Employee has a service appointment scheduled

	      Input: Selects an appointment to cancel

	      Expected Output: The appointment booking is cancelled and removed from the customer and shop
	      owner/employee schedules

	      Results: Passed

	\item \textbf{FRT-BE7-1}

	      Initial State: Shop Owner selects appointment availability

	      Input: Sets appointment availabilities

	      Expected Output: The appointment availability is updated in the database

	      Results: Passed (shop hours)

\end{enumerate}

\section{Nonfunctional Requirements Evaluation}

\subsection{Usability}

\subsection{Performance}

\subsection{etc.}

\section{Unit Testing}

\section{Changes Due to Testing}

\wss{This section should highlight how feedback from the users and from
	the supervisor (when one exists) shaped the final product.  In particular
	the feedback from the Rev 0 demo to the supervisor (or to potential users)
	should be highlighted.}

\section{Automated Testing}

All automated testing was carried out by Jest, a JavaScript testing framework. Each module has its
own corresponding test file located within the ``test'' folder. The ``test'' folder has the same
structure as the ``src'' folder, creating a one-to-one mapping of a module and its corresponding
test. For example, the module ``src/server/services/userService.ts'' has its tests located in
``test/server/services/userService.ts''. There are two Jest test runs, with the only difference
being whether the testing environment is connected to a real database or not (i.e., whether the
database needs to be seeded or mocked in each test case). As part of the CI/CD pipeline, the full
test suite is run and information regarding pass/fail and code coverage is reported on every pull
request to the main branch and to every push to the main branch.

\section{Trace to Requirements}

\section{Trace to Modules}

\section{Code Coverage Metrics}

\bibliographystyle{plainnat}

\bibliography{../../refs/References}

\newpage{}

\section{Appendix}
\subsection{Reflection}

The information in this section will be used to evaluate the team members on the graduate attribute
of Reflection. Please answer the following question:

\begin{enumerate}
	\item In what ways was the Verification and Validation (VnV) Plan different from the activities that were
	      actually conducted for VnV? If there were differences, what changes required the modification in
	      the plan? Why did these changes occur? Would you be able to anticipate these changes in future
	      projects? If there weren't any differences, how was your team able to clearly predict a feasible
	      amount of effort and the right tasks needed to build the evidence that demonstrates the required
	      quality? (It is expected that most teams will have had to deviate from their original VnV Plan.)
\end{enumerate}

\end{document}
