\documentclass[12pt, titlepage]{article}

\usepackage{booktabs}
\usepackage{tabularx}
\usepackage{hyperref}
\hypersetup{
    colorlinks,
    citecolor=black,
    filecolor=black,
    linkcolor=red,
    urlcolor=blue
}
\usepackage[round]{natbib}

%% Comments

\usepackage{color}

\newif\ifcomments\commentstrue %displays comments
%\newif\ifcomments\commentsfalse %so that comments do not display

\ifcomments
\newcommand{\authornote}[3]{\textcolor{#1}{[#3 ---#2]}}
\newcommand{\todo}[1]{\textcolor{red}{[TODO: #1]}}
\else
\newcommand{\authornote}[3]{}
\newcommand{\todo}[1]{}
\fi

\newcommand{\wss}[1]{\authornote{blue}{SS}{#1}} 
\newcommand{\plt}[1]{\authornote{magenta}{TPLT}{#1}} %For explanation of the template
\newcommand{\an}[1]{\authornote{cyan}{Author}{#1}}

%% Common Parts

\newcommand{\progname}{Sayyara}
\newcommand{\authname}{Team 3, Tiny Coders
	\\ Arkin Modi
	\\ Joy Xiao
	\\ Leon So
	\\ Timothy Choy} % AUTHOR NAMES

\usepackage{hyperref}
\hypersetup{colorlinks=true, linkcolor=blue, citecolor=blue, filecolor=blue,
	urlcolor=blue, unicode=false}
\urlstyle{same}

\usepackage{parskip}
\usepackage{geometry}
\geometry{a4paper, portrait, margin=1in}


\begin{document}

\title{Verification and Validation Report: \progname}
\author{\authname}
\date{\today}

\maketitle

\pagenumbering{roman}

\section{Revision History}

\begin{table}[hp]
	\caption{Revision History} \label{TblRevisionHistory}
	\begin{tabularx}{\textwidth}{llX}
		\toprule
		\textbf{Date}     & \textbf{Developer(s)} & \textbf{Change}                                              \\
		\midrule
		February 22, 2023 & Arkin Modi            & Add Automated Testing Section                                \\
		March 3, 2023     & Joy Xiao              & Add Tests Evaluations for Legal and Cultural Requirements    \\
		March 4, 2023     & Leon So               & Add Introduction                                             \\
		March 4, 2023     & Arkin Modi            & Add Tests Evaluation for Security Requirements               \\
		March 4, 2023     & Arkin Modi            & Update Look and Feel Requirements Tests                      \\
		March 4, 2023     & Leon So               & Add NFR Tests for Operational and Environmental Requirements \\
		March 4, 2023     & Timothy Choy          & Add Test Evaluations for Usability and Humanity Requirements \\
		\bottomrule
	\end{tabularx}
\end{table}
\newpage

\section{Symbols, Abbreviations and Acronyms}

\renewcommand{\arraystretch}{1.2}
\begin{tabular}{l l}
	\toprule
	\textbf{symbol} & \textbf{description}                \\
	\midrule
	API             & Application Programming Interface   \\
	CD              & Continuous Deployment               \\
	CI              & Continuous Integration              \\
	PII             & Personal Identifiable Information   \\
	PWA             & Progressive Web Application         \\
	SRS             & Software Requirements Specification \\
	T               & Test                                \\
	VnV             & Verification and Validation         \\
	\bottomrule
\end{tabular}\\

\newpage

\tableofcontents

\listoftables %if appropriate

\listoffigures %if appropriate

\newpage

\pagenumbering{arabic}

\section{Introduction}

\subsection{Project Summary}

Sayyara is a Progressive Web Application (PWA) which acts as a single platform for independent auto
repair shops and vehicle owners. This platform allows independent auto repair shops and vehicle
owners to interact in various ways. Using Sayyara, vehicle owners can search for auto repair shops
and services based on a variety of search filters; request quotes for service; book, view, and
manage service appointments. On the application, auto repair shop owners have full shop management
capabilities such as: adding and managing a list of employees; managing a list of service types and
corresponding service appointment availabilities; managing store information such as location,
hours of operation, and contact information. Auto repair shop owners and employees will be able to
manage quotes, service appointments, and work orders from a single application.

\subsection{Purpose of Document}
This document outlines the validation and verification process for the application, Sayyara, and a
detailed summary of results for the planned tests defined in the
\href{https://github.com/arkinmodi/project-sayyara/blob/main/docs/VnVPlan/VnVPlan.pdf}{VnV Plan}.

\subsection{Scope of Testing}
As defined in the
\href{https://github.com/arkinmodi/project-sayyara/blob/main/docs/VnVPlan/VnVPlan.pdf}{VnV Plan},
the tests reported aim to verify and validate functional and nonfunctional requirements listed in
the \href{https://github.com/arkinmodi/project-sayyara/blob/main/docs/SRS/SRS.pdf}{SRS}.

In addition, the testing reported in this document aims to validate that the application provides
adequate usability and to verify that system is in a functional state for the end users. The
testing and validation will aid in ensuring that the product fulfills the system requirements,
intended use, and goals of stakeholders.

\section{Functional Requirements Evaluation}

\section{Nonfunctional Requirements Evaluation}

\subsection{Look and Feel Requirements}
\begin{enumerate}
	\item \textbf{NFRT-LF1-1}

	      Type: Dynamic, Manual

	      Initial State: The application is accessible through the Google Chrome web browser

	      Input/Condition: The window size is changed

	      Output/Result: The application shall adjust and scale to fit the new window size

	      How test will be performed: The tester will access the application through Google Chrome on their
	      desktop/laptop and change the windows through the use of Google Chrome DevTools Device Toolbar

	      Results: Passed

	\item \textbf{NFRT-LF2-1}

	      Type: Dynamic, Manual

	      Initial State: The application is accessible through the web browser

	      Input: The application is opened in a full screened web browser window

	      Output: All text on the screen shall be readable

	      How test will be performed: The tester open the application in a full screened web browser window,
	      navigate to all pages, and judge if all text on the screen is readable from sitting 50 centimeters
	      away from the monitor

	      Results: Passed

	\item \textbf{NFRT-LF3-1}

	      Type: Dynamic, Manual

	      Initial State: The application is accessible through the web browser and there is a completed work
	      order and quote on the user's account

	      Input: The user opens the work order details and the quote details

	      Output: All currency shall be rounded to two decimal places

	      How test will be performed: The tester will navigate to the work order and quote and verify that
	      all values of currency are rounded to two decimal places

	      Results: Passed

\end{enumerate}

\subsection{Usability and Humanity Requirements}
\begin{enumerate}
	\item \textbf{NFRT-UH1-1}

	      Type: Dynamic, Manual

	      How test will be performed: The testers will complete the manual system tests for functional
	      requirements on a MacOS desktop/laptop device, a Windows desktop/laptop, an iOS mobile device, and
	      an android mobile device.

	      Results: Passed. Testers were asked to write down which device they used to test to ensure all
	      devices and operating systems were covered.

	\item \textbf{NFRT-UH2-1}

	      Type: Dynamic, Manual

	      Initial State: Device is connected to the internet and application is not open.

	      Input/Condition: The user launches the application on a web browser.

	      Output/Result: The system can be assessed through the web browser.

	      How test will be performed: The testers will attempt to launch the application on a web browser,
	      using a device that is connected to the internet.

	      Results: Passed. The testers were able to access the application using the web browser while
	      connected to the internet.

	\item \textbf{NFRT-UH3-1}

	      Type: Dynamic, Manual

	      Initial State: Device is connected to the internet and application is open.

	      Input/Condition: User disconnects from the internet.

	      Output/Result: The system notifies the user that there is no network connection.

	      How test will be performed: The testers disconnects from the internet while the application is
	      open.

	      Results: Passed. Testers were greeted with a toast when they disconnected from the internet.

\end{enumerate}

\subsection{Performance Requirements}

\subsection{Operational and Environmental Requirements}
\begin{enumerate}
	\item \textbf{NFRT-OE-1}

	      Type: Dynamic, Manual

	      Initial State: The application is accessible through the Google Chrome web browser

	      Input: The user navigates across all pages and dialogs, and performs all manual system tests for
	      functional requirements

	      Output: All pages and dialogs should be fully operational on the below listed device dimensions

	      How test will be performed: The testers will complete the manual system tests for functional
	      requirements on Google Chrome using various device dimension options listed in the Google DevTools
	      Device Toolbar. At minimum, the testers will test with the following dimensions: Responsive, iPhone
	      12 Pro ($390 \times 844$), iPhone SE ($375 \times 667$), and iPad Air ($820 \times 1180$). This set
	      will provide a reasonable variety of different device dimensions.

	      Results: Failed. Manually tested to verify whether the application supports the above device
	      dimensions. Shop lookup page does not work on iPad Air ($820 \times 1180$) dimensions. Chat window
	      does not work on iPhone SE ($375 \times 667$) dimensions.
\end{enumerate}

\subsection{Maintainability and Support Requirements}

\subsection{Security Requirements}
\begin{enumerate}

	\item \textbf{NFRT-SR1-1}

	      Type: Dynamic, Automatic

	      Initial State: An account with a created appointment

	      Input/Condition: An HTTP request to the application's backend to get the appointment by appointment
	      ID

	      Output/Result: The request is rejected with a 403 Forbidden error message

	      How test will be performed: Through Jest an HTTP request will be sent to the server

	      Results: Passed.

	\item \textbf{NFRT-SR2-1}

	      Type: Dynamic, Manual

	      Initial State: One Shop Owner account and two Customer accounts with one customer having an
	      appointment at the shop registered under the same Shop Owner account

	      Input/Condition: Request available time slots at the Shop Owner's store using the Customer without
	      an appointment's account

	      Output/Result: The Customer should only be able to view the time slot taken by the other customer
	      and no other extra information

	      How test will be performed: The tester will view the available time slots page as the Customer
	      without an appointment

	      Results: Failed. Extra information is returned from the API call that is not needed for this
	      endpoints and exposes data that does not belong to user. None of the data contains PII.

\end{enumerate}

\subsection{Cultural Requirements}
\begin{enumerate}
	\item \textbf{NFRT-CR1-1}

	      Type: Static, Manual

	      Initial State: The application is accessible through the web browser

	      Input: The user navigates across all pages and dialogs, and performs all end-to-end tests

	      Output: There should be no offensive texts or images

	      How test will be performed: The testers will go through end-to-end tests for the entire program and
	      will confirm that any texts or images are not offensive.

	      Results: Passed. Manually tested to verify that there are no offensive images or text.

	\item \textbf{NFRT-CR2-1}
	      Type: Static, Manual

	      Initial State: The application is accessible through the web browser

	      Input: The user navigates across all pages and dialogs, and performs all end-to-end tests

	      Output: All text should be displayed in English

	      How test will be performed: The testers will go through end-to-end tests for the entire program and
	      will confirm that the text displayed is in English.

	      Results: Passed. Checked that everything is written in English in the implementation,
	      documentation, and anything visible to the user.

\end{enumerate}

\subsection{Legal Requirements}
\begin{enumerate}
	\item \textbf{NFRT-LR1-1}

	      Type: Static, Manual

	      Input: The user navigates all source code and project documents

	      Output: Appropriate credits and/or copyright licenses are used, and MIT License requirements should
	      be adhered to

	      How test will be performed: The testers will go through the source code of the project and project
	      documents to determine if any open source resources were used and check that appropriate credits
	      and/or copyright licenses are used.

	      Results: Passed. Checked the MIT License requirements and that the project adhered to all the
	      requirements.

\end{enumerate}

\section{Unit Testing}

\section{Changes Due to Testing}

\wss{This section should highlight how feedback from the users and from
	the supervisor (when one exists) shaped the final product.  In particular
	the feedback from the Rev 0 demo to the supervisor (or to potential users)
	should be highlighted.}

\section{Automated Testing}

All automated testing was carried out by Jest, a JavaScript testing framework. Each module has its
own corresponding test file located within the ``test'' folder. The ``test'' folder has the same
structure as the ``src'' folder, creating a one-to-one mapping of a module and its corresponding
test. For example, the module ``src/server/services/userService.ts'' has its tests located in
``test/server/services/userService.ts''. There are two Jest test runs, with the only difference
being whether the testing environment is connected to a real database or not (i.e., whether the
database needs to be seeded or mocked in each test case). As part of the CI/CD pipeline, the full
test suite is run and information regarding pass/fail and code coverage is reported on every pull
request to the main branch and to every push to the main branch.

\section{Trace to Requirements}

\section{Trace to Modules}

\section{Code Coverage Metrics}

\bibliographystyle{plainnat}

\bibliography{../../refs/References}

\newpage{}

\section{Appendix}
\subsection{Reflection}

The information in this section will be used to evaluate the team members on the graduate attribute
of Reflection. Please answer the following question:

\begin{enumerate}
	\item In what ways was the Verification and Validation (VnV) Plan different from the activities that were
	      actually conducted for VnV? If there were differences, what changes required the modification in
	      the plan? Why did these changes occur? Would you be able to anticipate these changes in future
	      projects? If there weren't any differences, how was your team able to clearly predict a feasible
	      amount of effort and the right tasks needed to build the evidence that demonstrates the required
	      quality? (It is expected that most teams will have had to deviate from their original VnV Plan.)
\end{enumerate}

\end{document}
