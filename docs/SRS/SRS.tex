\documentclass[12pt]{article}

\usepackage{amsmath, mathtools}
\usepackage{amsfonts}
\usepackage{amssymb}
\usepackage{graphicx}
\usepackage{colortbl}
\usepackage{xr}
\usepackage{hyperref}
\usepackage{longtable}
\usepackage{xfrac}
\usepackage{tabularx}
\usepackage{float}
\usepackage{siunitx}
\usepackage{booktabs}
\usepackage{caption}
\usepackage{pdflscape}
\usepackage{afterpage}

\usepackage[round]{natbib}

%\usepackage{refcheck}

\hypersetup{
    bookmarks=true,         % show bookmarks bar?
    colorlinks=true,        % false: boxed links; true: colored links
    linkcolor=red,          % color of internal links (change box color with linkbordercolor)
    citecolor=green,        % color of links to bibliography
    filecolor=magenta,      % color of file links
    urlcolor=cyan           % color of external links
}

%% Comments

\usepackage{color}

\newif\ifcomments\commentstrue %displays comments
%\newif\ifcomments\commentsfalse %so that comments do not display

\ifcomments
\newcommand{\authornote}[3]{\textcolor{#1}{[#3 ---#2]}}
\newcommand{\todo}[1]{\textcolor{red}{[TODO: #1]}}
\else
\newcommand{\authornote}[3]{}
\newcommand{\todo}[1]{}
\fi

\newcommand{\wss}[1]{\authornote{blue}{SS}{#1}} 
\newcommand{\plt}[1]{\authornote{magenta}{TPLT}{#1}} %For explanation of the template
\newcommand{\an}[1]{\authornote{cyan}{Author}{#1}}

%% Common Parts

\newcommand{\progname}{Sayyara}
\newcommand{\authname}{Team 3, Tiny Coders
	\\ Arkin Modi
	\\ Joy Xiao
	\\ Leon So
	\\ Timothy Choy} % AUTHOR NAMES

\usepackage{hyperref}
\hypersetup{colorlinks=true, linkcolor=blue, citecolor=blue, filecolor=blue,
	urlcolor=blue, unicode=false}
\urlstyle{same}

\usepackage{parskip}
\usepackage{geometry}
\geometry{a4paper, portrait, margin=1in}


% For easy change of table widths
\newcommand{\colZwidth}{1.0\textwidth}
\newcommand{\colAwidth}{0.13\textwidth}
\newcommand{\colBwidth}{0.82\textwidth}
\newcommand{\colCwidth}{0.1\textwidth}
\newcommand{\colDwidth}{0.05\textwidth}
\newcommand{\colEwidth}{0.8\textwidth}
\newcommand{\colFwidth}{0.17\textwidth}
\newcommand{\colGwidth}{0.5\textwidth}
\newcommand{\colHwidth}{0.28\textwidth}

% Used so that cross-references have a meaningful prefix
\newcounter{defnum} %Definition Number
\newcommand{\dthedefnum}{GD\thedefnum}
\newcommand{\dref}[1]{GD\ref{#1}}
\newcounter{datadefnum} %Data definition Number
\newcommand{\ddthedatadefnum}{DD\thedatadefnum}
\newcommand{\ddref}[1]{DD\ref{#1}}
\newcounter{theorynum} %Theory Number
\newcommand{\tthetheorynum}{T\thetheorynum}
\newcommand{\tref}[1]{T\ref{#1}}
\newcounter{tablenum} %Table Number
\newcommand{\tbthetablenum}{T\thetablenum}
\newcommand{\tbref}[1]{TB\ref{#1}}
\newcounter{assumpnum} %Assumption Number
\newcommand{\atheassumpnum}{P\theassumpnum}
\newcommand{\aref}[1]{A\ref{#1}}
\newcounter{goalnum} %Goal Number
\newcommand{\gthegoalnum}{P\thegoalnum}
\newcommand{\gsref}[1]{GS\ref{#1}}
\newcounter{instnum} %Instance Number
\newcommand{\itheinstnum}{IM\theinstnum}
\newcommand{\iref}[1]{IM\ref{#1}}
\newcounter{reqnum} %Requirement Number
\newcommand{\rthereqnum}{P\thereqnum}
\newcommand{\rref}[1]{R\ref{#1}}
\newcounter{nfrnum} %NFR Number
\newcommand{\rthenfrnum}{NFR\thenfrnum}
\newcommand{\nfrref}[1]{NFR\ref{#1}}
\newcounter{lcnum} %Likely change number
\newcommand{\lthelcnum}{LC\thelcnum}
\newcommand{\lcref}[1]{LC\ref{#1}}

\usepackage{fullpage}

\newcommand{\deftheory}[9][Not Applicable]
{
\newpage
\noindent \rule{\textwidth}{0.5mm}

\paragraph{RefName: } \textbf{#2} \phantomsection 
\label{#2}

\paragraph{Label:} #3

\noindent \rule{\textwidth}{0.5mm}

\paragraph{Equation:}

#4

\paragraph{Description:}

#5

\paragraph{Notes:}

#6

\paragraph{Source:}

#7

\paragraph{Ref.\ By:}

#8

\paragraph{Preconditions for \hyperref[#2]{#2}:}
\label{#2_precond}

#9

\paragraph{Derivation for \hyperref[#2]{#2}:}
\label{#2_deriv}

#1

\noindent \rule{\textwidth}{0.5mm}

}

\begin{document}

\title{Software Requirements Specification for \progname: subtitle describing software}
\author{\authname}
\date{\today}

\maketitle

~\newpage

\pagenumbering{roman}

\tableofcontents

~\newpage

\begin{table}[hp]
	\caption{Revision History} \label{TblRevisionHistory}
	\begin{tabularx}{\textwidth}{llX}
		\toprule
		\textbf{Date}      & \textbf{Developer(s)} & \textbf{Change}                                                                    \\
		\midrule
		September 30, 2022 & Leon So               & Add purpose of project                                                             \\
		September 30, 2022 & Arkin Modi            & Add open issues and new problems sections (effects on the current environment)     \\
		October 1, 2022    & Arkin Modi            & Add user documentation and training, waiting room and ideas for solutions sections \\
		October 1, 2022    & Arkin Modi            & Add migration to the new product, risks, and costs sections                        \\
		\bottomrule
	\end{tabularx}
\end{table}

\newpage

\pagenumbering{arabic}

This document describes the requirements for .... The template for the Software Requirements
Specification (SRS) is a subset of the Volere template~\citep{RobertsonAndRobertson2012}. If you
make further modifications to the template, you should explicitly state what modifications were
made.

\section{Project Drivers}

\subsection{The Purpose of the Project}

Independent auto repair shops do not have an efficient way of reaching and interacting with new
customers. Currently, many independent shop owners rely on word-of-mouth referrals as a main
channel to acquiring new customers. Independent auto repair shops are also spending a significant
amount of their time on administrative work such as managing appointments and providing quotes. As
a result, independent auto repair shops have a difficult time competing with larger repair shops
which have dedicated systems and services in place.

On the other hand, customers do not have an effective way to find and compare auto repair shops.
Currently, one of the only ways to compare repair shops is by manually searching or reaching out to
repair shops one-by-one. This process can often be repetitive and time-consuming.

Sayyara is a progressive web application (PWA) which will act as a single platform for independent
auto repair shops and vehicle owners. This platform will allow independent auto repair shops and
vehicle owners to interact in a more efficient and effective manner. Vehicle owners can search for
auto repair shops and services based on a variety of search filters; request quotes for service;
book, view, and manage service appointments. On the application, auto repair shop owners will have
full shop management capabilities such as: adding and managing a list of employees; managing a list
of service types and corresponding service appointment availabilities; managing store information
such as location, hours of operation, and contact information. Auto repair shop owners and
employees will be able to manage quotes, service appointments, and work orders from a single
application. Ultimately, Sayyara will significantly improve the auto repair experience for both
independent auto repair shops and vehicle owners.

\subsection{The Stakeholders}

\subsubsection{The Client}

\subsubsection{The Customers}

\subsubsection{Other Stakeholders}

\subsection{Mandated Constraints}

\subsection{Naming Conventions and Terminology}

\subsection{Relevant Facts and Assumptions}

User characteristics should go under assumptions.

\section{Functional Requirements}

\subsection{The Scope of the Work and the Product}

\subsubsection{The Context of the Work}

\subsubsection{Work Partitioning}

\subsubsection{Individual Product Use Cases}

\subsection{Functional Requirements}

\section{Non-functional Requirements}

\subsection{Look and Feel Requirements}

\subsection{Usability and Humanity Requirements}

\subsection{Performance Requirements}

\subsection{Operational and Environmental Requirements}

\subsection{Maintainability and Support Requirements}

\subsection{Security Requirements}

\subsection{Cultural Requirements}

\subsection{Legal Requirements}

\subsection{Health and Safety Requirements}

This section is not in the original Volere template, but health and safety are issues that should
be considered for every engineering project.

\section{Project Issues}

\subsection{Open Issues}

There are currently no known open issues that may lead to significant change to the product or its
design.

\subsection{Off-the-Shelf Solutions}

\subsubsection{Ready-Made Products}
\subsubsection{Resuable Components}
\subsubsection{Products That Can Be Copied}

\subsection{New Problems}

\subsubsection{Effects on the Current Environment}

This application will change the way certain processes are preformed and these changes will impact
the users.

\textbf{Work Orders}

The work order system will affect the way automotive mechanics document their work. The data will
be inputted into the application therefore any failures can result in data loss.

\textbf{Appointments}

% todo: what can happen if failure?
The appointments system will affect the way that both the customers and the receptionists schedule
appointments. The application will track daily appointment schedules and report time conflicts. The
application shall not lock the receptionist out of overriding the schedule.

\textbf{Searching}

% todo: add what has changed?, what the system should not do?, and what can happen if failure?
The work order system will affect the way customers search for automotive repair shops.

\textbf{Quotes}

% todo: add what has changed?, what the system should not do?, and what can happen if failure?
The appointments system will affect the way that both the customers and automotive repair shops
communicate quotes.

\subsubsection{Effects on the Installed Systems}

\subsubsection{Potential User Problems}

\subsubsection{Limitations in the Anticipated Implementation Environment That May Inhibit the New Product}

\subsubsection{Follow-Up Problems}

\subsection{Tasks}
\subsubsection{Project Planning}
\subsubsection{Planning of the Development Phases}

\subsection{Migration to the New Product}
\subsubsection{Requirements for Migration to the New Product}

There are no requirements for migrating to the new product.

\subsubsection{Data That Has to Be Modified or Translated for the New System}

No data needs to be modified or translated to the new system.

\subsection{Risks}

\begin{itemize}
	\item Failures in the work orders and quotes work flow may lead to data loss.
	\item Failures in the appointments work flow may lead to a loss in appointment or conflicting
	      appointments.
	\item Failure to meet deadlines will cause set backs in project's timeline. In the event of this, lower
	      priority requirements may need to be dropped.
\end{itemize}

\subsection{Costs}

There are no financial costs associated with the development of this application. All software and
cloud infrastructure used are free to use. There will be about six months of development time
required.

\subsection{User Documentation and Training}
\subsubsection{User Documentation Requirements}

The application will feature a ``Getting Started'' guide, where it shall guide the user through the
most common use cases. For vehicle owners, the use cases will include: searching for shops,
requesting quotes, and scheduling appointments. For automotive shops, the use cases include:
setting shop details, managing appointments, managing employees, responding to quotes, and managing
work orders.

\subsubsection{Training Requirements}

Knowledge of how to navigate a website will be required. Documentation concerning detailed usage of
the website's user flows will be provided to the user.

\subsection{Waiting Room}

There are currently no requirements that are not part of the initial release.

% todo: could be a good place to put the stretch goals

\subsection{Ideas for Solutions}

% todo: ask the rest of the team on what they would add here

During the requirements collection and understanding phase, some ideas with regard to what tooling
will be used have occurred.

\begin{itemize}
	\item Form
	      \begin{itemize}
		      \item With the constraint of this application to be a PWA, the idea of using a React based framework,
		            specifically Next.js.
	      \end{itemize}
	\item Authentication
	      \begin{itemize}
		      \item To handle authentication, using emails and passwords, with the package NextAuth.js.
		      \item Creating a dedicated endpoint in the backend for looking up user information.
	      \end{itemize}
\end{itemize}

\newpage

\bibliographystyle{plainnat}

\bibliography {../../refs/References}

\newpage

\noindent \plt{The following is not part of the template, just some things to consider
	when filing in the template.}

\noindent \plt{Grammar, flow and \LaTeX advice:
	\begin{itemize}
		\item For Mac users \texttt{*.DS\_Store} should be in \texttt{.gitignore}
		\item \LaTeX{} and formatting rules
		      \begin{itemize}
			      \item Variables are italic, everything else not, includes subscripts (link to document)
			            \begin{itemize}
				            \item \href{https://physics.nist.gov/cuu/pdf/typefaces.pdf}{Conventions}
				            \item Watch out for implied multiplication
			            \end{itemize}
			      \item Use BibTeX
			      \item Use cross-referencing
		      \end{itemize}
		\item Grammar and writing rules
		      \begin{itemize}
			      \item Acronyms expanded on first usage (not just in table of acronyms)
			      \item ``In order to'' should be ``to''
		      \end{itemize}
	\end{itemize}}

\noindent \plt{Advice on using the template:
	\begin{itemize}
		\item Difference between physical and software constraints
		\item Properties of a correct solution means \emph{additional} properties, not a restating of the
		      requirements (may be ``not applicable'' for your problem). If you have a table of output
		      constraints, then these are properties of a correct solution.
		\item Assumptions have to be invoked somewhere
		\item ``Referenced by'' implies that there is an explicit reference
		\item Think of traceability matrix, list of assumption invocations and list of reference by fields as
		      automatically generatable
		\item If you say the format of the output (plot, table etc), then your requirement could be more abstract
	\end{itemize}
}
\section{Appendix}

This section has been added to the Volere template. This is where you can place additional
information.

\newpage{}
\section*{Appendix --- Reflection}

The information in this section will be used to evaluate the team members on the graduate attribute
of Lifelong Learning. Please answer the following questions:

\begin{enumerate}
	\item What knowledge and skills will the team collectively need to acquire to successfully complete this
	      capstone project? Examples of possible knowledge to acquire include domain specific knowledge from
	      the domain of your application, or software engineering knowledge, mechatronics knowledge or
	      computer science knowledge. Skills may be related to technology, or writing, or presentation, or
	      team management, etc. You should look to identify at least one item for each team member.
	\item For each of the knowledge areas and skills identified in the previous question, what are at least
	      two approaches to acquiring the knowledge or mastering the skill? Of the identified approaches,
	      which will each team member pursue, and why did they make this choice?
\end{enumerate}

\subsection{Symbolic Parameters}

The definition of the requirements will likely call for SYMBOLIC\_CONSTANTS. Their values are
defined in this section for easy maintenance.

\end{document}
