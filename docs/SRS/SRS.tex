\documentclass[12pt]{article}

\usepackage{amsmath, mathtools}
\usepackage{amsfonts}
\usepackage{amssymb}
\usepackage{graphicx}
\usepackage{colortbl}
\usepackage{xr}
\usepackage{hyperref}
\usepackage{longtable}
\usepackage{xfrac}
\usepackage{tabularx}
\usepackage{float}
\usepackage{siunitx}
\usepackage{booktabs}
\usepackage{caption}
\usepackage{pdflscape}
\usepackage{afterpage}

\usepackage[round]{natbib}

%\usepackage{refcheck}

\hypersetup{
    bookmarks=true,         % show bookmarks bar?
    colorlinks=true,        % false: boxed links; true: colored links
    linkcolor=red,          % color of internal links (change box color with linkbordercolor)
    citecolor=green,        % color of links to bibliography
    filecolor=magenta,      % color of file links
    urlcolor=cyan           % color of external links
}

%% Comments

\usepackage{color}

\newif\ifcomments\commentstrue %displays comments
%\newif\ifcomments\commentsfalse %so that comments do not display

\ifcomments
\newcommand{\authornote}[3]{\textcolor{#1}{[#3 ---#2]}}
\newcommand{\todo}[1]{\textcolor{red}{[TODO: #1]}}
\else
\newcommand{\authornote}[3]{}
\newcommand{\todo}[1]{}
\fi

\newcommand{\wss}[1]{\authornote{blue}{SS}{#1}} 
\newcommand{\plt}[1]{\authornote{magenta}{TPLT}{#1}} %For explanation of the template
\newcommand{\an}[1]{\authornote{cyan}{Author}{#1}}

%% Common Parts

\newcommand{\progname}{Sayyara}
\newcommand{\authname}{Team 3, Tiny Coders
	\\ Arkin Modi
	\\ Joy Xiao
	\\ Leon So
	\\ Timothy Choy} % AUTHOR NAMES

\usepackage{hyperref}
\hypersetup{colorlinks=true, linkcolor=blue, citecolor=blue, filecolor=blue,
	urlcolor=blue, unicode=false}
\urlstyle{same}

\usepackage{parskip}
\usepackage{geometry}
\geometry{a4paper, portrait, margin=1in}


% For easy change of table widths
\newcommand{\colZwidth}{1.0\textwidth}
\newcommand{\colAwidth}{0.13\textwidth}
\newcommand{\colBwidth}{0.82\textwidth}
\newcommand{\colCwidth}{0.1\textwidth}
\newcommand{\colDwidth}{0.05\textwidth}
\newcommand{\colEwidth}{0.8\textwidth}
\newcommand{\colFwidth}{0.17\textwidth}
\newcommand{\colGwidth}{0.5\textwidth}
\newcommand{\colHwidth}{0.28\textwidth}

% Used so that cross-references have a meaningful prefix
\newcounter{defnum} %Definition Number
\newcommand{\dthedefnum}{GD\thedefnum}
\newcommand{\dref}[1]{GD\ref{#1}}
\newcounter{datadefnum} %Data definition Number
\newcommand{\ddthedatadefnum}{DD\thedatadefnum}
\newcommand{\ddref}[1]{DD\ref{#1}}
\newcounter{theorynum} %Theory Number
\newcommand{\tthetheorynum}{T\thetheorynum}
\newcommand{\tref}[1]{T\ref{#1}}
\newcounter{tablenum} %Table Number
\newcommand{\tbthetablenum}{T\thetablenum}
\newcommand{\tbref}[1]{TB\ref{#1}}
\newcounter{assumpnum} %Assumption Number
\newcommand{\atheassumpnum}{P\theassumpnum}
\newcommand{\aref}[1]{A\ref{#1}}
\newcounter{goalnum} %Goal Number
\newcommand{\gthegoalnum}{P\thegoalnum}
\newcommand{\gsref}[1]{GS\ref{#1}}
\newcounter{instnum} %Instance Number
\newcommand{\itheinstnum}{IM\theinstnum}
\newcommand{\iref}[1]{IM\ref{#1}}
\newcounter{reqnum} %Requirement Number
\newcommand{\rthereqnum}{P\thereqnum}
\newcommand{\rref}[1]{R\ref{#1}}
\newcounter{nfrnum} %NFR Number
\newcommand{\rthenfrnum}{NFR\thenfrnum}
\newcommand{\nfrref}[1]{NFR\ref{#1}}
\newcounter{lcnum} %Likely change number
\newcommand{\lthelcnum}{LC\thelcnum}
\newcommand{\lcref}[1]{LC\ref{#1}}

\usepackage{fullpage}

\newcommand{\deftheory}[9][Not Applicable]
{
\newpage
\noindent \rule{\textwidth}{0.5mm}

\paragraph{RefName: } \textbf{#2} \phantomsection 
\label{#2}

\paragraph{Label:} #3

\noindent \rule{\textwidth}{0.5mm}

\paragraph{Equation:}

#4

\paragraph{Description:}

#5

\paragraph{Notes:}

#6

\paragraph{Source:}

#7

\paragraph{Ref.\ By:}

#8

\paragraph{Preconditions for \hyperref[#2]{#2}:}
\label{#2_precond}

#9

\paragraph{Derivation for \hyperref[#2]{#2}:}
\label{#2_deriv}

#1

\noindent \rule{\textwidth}{0.5mm}

}

\begin{document}

\title{Software Requirements Specification for \progname: subtitle describing software}
\author{\authname}
\date{\today}

\maketitle

~\newpage

\pagenumbering{roman}

\tableofcontents

~\newpage

\begin{table}[hp]
	\caption{Revision History} \label{TblRevisionHistory}
	\begin{tabularx}{\textwidth}{llX}
		\toprule
		\textbf{Date}      & \textbf{Developer(s)} & \textbf{Change}                                \\
		\midrule
		September 30, 2022 & Leon So               & Add purpose of project                         \\
		September 30, 2022 & Joy Xiao              & Add stakeholders                               \\
		September 30, 2022 & Leon So               & Add functional requirements for authentication \\
		\bottomrule
	\end{tabularx}
\end{table}

\newpage

\pagenumbering{arabic}

This document describes the requirements for .... The template for the Software Requirements
Specification (SRS) is a subset of the Volere template~\citep{RobertsonAndRobertson2012}. If you
make further modifications to the template, you should explicity state what modifications were
made.

\section{Project Drivers}

\subsection{The Purpose of the Project}

Independent auto repair shops do not have an efficient way of reaching and interacting with new
customers. Currently, many independent shop owners rely on word-of-mouth referrals as a main
channel to acquiring new customers. Independent auto repair shops are also spending a significant
amount of their time on administrative work such as managing appointments and providing quotes. As
a result, independent auto repair shops have a difficult time competing with larger repair shops
which have dedicated systems and services in place.

On the other hand, customers do not have an effective way to find and compare auto repair shops.
Currently, one of the only ways to compare repair shops is by manually searching or reaching out to
repair shops one-by-one. This process can often be repetitive and time-consuming.

Sayyara is a progressive web application (PWA) which will act as a single platform for independent
auto repair shops and vehicle owners. This platform will allow independent auto repair shops and
vehicle owners to interact in a more efficient and effective manner. Vehicle owners can search for
auto repair shops and services based on a variety of search filters; request quotes for service;
book, view, and manage service appointments. On the application, auto repair shop owners will have
full shop management capabilities such as: adding and managing a list of employees; managing a list
of service types and corresponding service appointment availabilities; managing store information
such as location, hours of operation, and contact information. Auto repair shop owners and
employees will be able to manage quotes, service appointments, and work orders from a single
application. Ultimately, Sayyara will significantly improve the auto repair experience for both
independent auto repair shops and vehicle owners.

\subsection{The Stakeholders}

\subsubsection{The Client}
The client of the project is Nabeel Ibrahim. Nabeel will be the point of contact throughout the
development of the project.

\subsubsection{The Customers}
The customers of Sayyara will be independent auto repair shop owners, shop employees, and vehicle
owners who are looking for a vehicle repair or maintenance service.

\subsubsection{Other Stakeholders}
Other stakeholders of the project are the developers, Tiny Coders, who are designing and
implementing the project.

\subsection{Mandated Constraints}

\subsection{Naming Conventions and Terminology}

\subsection{Relevant Facts and Assumptions}

User characteristics should go under assumptions.

\section{Functional Requirements}

\subsection{The Scope of the Work and the Product}

\subsubsection{The Context of the Work}

\subsubsection{Work Partitioning}

\subsubsection{Individual Product Use Cases}

\subsection{Functional Requirements}
\subsubsection{Authentication}
\begin{enumerate}[{BE}1.]
	\item The user wants to sign up for an account
	      \begin{enumerate}[{VP1}.1]
		      \item Viewpoint: Vehicle Owner
		            \begin{enumerate}
			            \item The system shall allow the user to enter an email and password
			            \item The system shall allow the user to enter their name
			            \item The system shall allow the user to enter their phone number
			            \item The system shall transition to the user landing page after the registration process is complete and
			                  successful
			            \item The system shall allow the user to cancel and exit the registration process
		            \end{enumerate}
	      \end{enumerate}
	      \begin{enumerate}[{VP1}.2]
		      \item Viewpoint: Auto Repair Shop Owner
		            \begin{enumerate}
			            \item The system shall allow the user to enter an email and password
			            \item The system shall allow the user to enter their name
			            \item The system shall allow the user to enter their phone number
			            \item The system shall allow the user to enter the shop name
			            \item \item The system shall allow the user to enter the shop address
			            \item The system shall allow the user to enter the shop phone number
			            \item The system shall transition to the shop owner landing page after the registration process is
			                  complete and successful
			            \item The system shall allow the user to cancel and exit the registration process
		            \end{enumerate}
	      \end{enumerate}
	      \begin{enumerate}[{VP1}.3]
		      \item Viewpoint: Auto Repair Shop Employee
		            \begin{enumerate}
			            \item The system shall allow the user to enter an email and password
			            \item The system shall allow the user to enter their name
			            \item The system shall allow the user to enter their phone number
			            \item The system shall transition to the employee landing page after the registration process is complete
			                  and successful
			            \item The system shall allow the user to cancel and exit the registration process
		            \end{enumerate}
	      \end{enumerate}
\end{enumerate}

\begin{enumerate}[{BE}2.]
	\item The user wants to login to their account
	      \begin{enumerate}[{VP1}.1]
		      \item Viewpoint: Vehicle Owner
		            \begin{enumerate}
			            \item The system shall allow the user to enter their email and password
			            \item The system shall transition to the user landing page after the login process is complete and
			                  successful
			            \item The system shall allow the user to cancel and exit the login process
		            \end{enumerate}
	      \end{enumerate}
	      \begin{enumerate}[{VP1}.2]
		      \item Viewpoint: Auto Repair Shop Owner
		            \begin{enumerate}
			            \begin{enumerate}
				            \item The system shall allow the user to enter their email and password
				            \item The system shall transition to the user landing page after the login process is complete and
				                  successful
				            \item The system shall allow the user to cancel and exit the login process
			            \end{enumerate}
		            \end{enumerate}
	      \end{enumerate}
	      \begin{enumerate}[{VP1}.3]
		      \item Viewpoint: Auto Repair Shop Employee
		            \begin{enumerate}
			            \begin{enumerate}
				            \item The system shall allow the user to enter their email and password
				            \item The system shall transition to the user landing page after the login process is complete and
				                  successful
				            \item The system shall allow the user to cancel and exit the login process
			            \end{enumerate}
		            \end{enumerate}
	      \end{enumerate}
\end{enumerate}

\section{Non-functional Requirements}

\subsection{Look and Feel Requirements}

\subsection{Usability and Humanity Requirements}

\subsection{Performance Requirements}

\subsection{Operational and Environmental Requirements}

\subsection{Maintainability and Support Requirements}

\subsection{Security Requirements}

\subsection{Cultural Requirements}

\subsection{Legal Requirements}

\subsection{Health and Safety Requirements}

This section is not in the original Volere template, but health and safety are issues that should
be considered for every engineering project.

\section{Project Issues}

\subsection{Open Issues}

\subsection{Off-the-Shelf Solutions}

\subsection{New Problems}

\subsection{Tasks}

\subsection{Migration to the New Product}

\subsection{Risks}

\subsection{Costs}

\subsection{User Documentation and Training}

\subsection{Waiting Room}

\subsection{Ideas for Solutions}

\newpage

\bibliographystyle{plainnat}

\bibliography {../../refs/References}

\newpage

\noindent \plt{The following is not part of the template, just some things to consider
	when filing in the template.}

\noindent \plt{Grammar, flow and \LaTeX advice:
	\begin{itemize}
		\item For Mac users \texttt{*.DS\_Store} should be in \texttt{.gitignore}
		\item \LaTeX{} and formatting rules
		      \begin{itemize}
			      \item Variables are italic, everything else not, includes subscripts (link to document)
			            \begin{itemize}
				            \item \href{https://physics.nist.gov/cuu/pdf/typefaces.pdf}{Conventions}
				            \item Watch out for implied multiplication
			            \end{itemize}
			      \item Use BibTeX
			      \item Use cross-referencing
		      \end{itemize}
		\item Grammar and writing rules
		      \begin{itemize}
			      \item Acronyms expanded on first usage (not just in table of acronyms)
			      \item ``In order to'' should be ``to''
		      \end{itemize}
	\end{itemize}}

\noindent \plt{Advice on using the template:
	\begin{itemize}
		\item Difference between physical and software constraints
		\item Properties of a correct solution means \emph{additional} properties, not a restating of the
		      requirements (may be ``not applicable'' for your problem). If you have a table of output
		      constraints, then these are properties of a correct solution.
		\item Assumptions have to be invoked somewhere
		\item ``Referenced by'' implies that there is an explicit reference
		\item Think of traceability matrix, list of assumption invocations and list of reference by fields as
		      automatically generatable
		\item If you say the format of the output (plot, table etc), then your requirement could be more abstract
	\end{itemize}
}
\section{Appendix}

This section has been added to the Volere template. This is where you can place additional
information.

\newpage{}
\section*{Appendix --- Reflection}

The information in this section will be used to evaluate the team members on the graduate attribute
of Lifelong Learning. Please answer the following questions:

\begin{enumerate}
	\item What knowledge and skills will the team collectively need to acquire to successfully complete this
	      capstone project? Examples of possible knowledge to acquire include domain specific knowledge from
	      the domain of your application, or software engineering knowledge, mechatronics knowledge or
	      computer science knowledge. Skills may be related to technology, or writing, or presentation, or
	      team management, etc. You should look to identify at least one item for each team member.
	\item For each of the knowledge areas and skills identified in the previous question, what are at least
	      two approaches to acquiring the knowledge or mastering the skill? Of the identified approaches,
	      which will each team member pursue, and why did they make this choice?
\end{enumerate}

\subsection{Symbolic Parameters}

The definition of the requirements will likely call for SYMBOLIC\_CONSTANTS. Their values are
defined in this section for easy maintenance.

\end{document}
